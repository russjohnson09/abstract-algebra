%Written using Texmaker.
%On linux 
%sudo apt-get install texlive-full
%sudo apt-get install texmaker
\documentclass[11pt]{article}
\usepackage[a4paper]{geometry} 
\geometry{a4paper, top=1.2cm, left=1.5cm, right=2.0cm, bottom=.1cm, includehead, includefoot}
\usepackage{amsmath}
\usepackage{amsfonts}
\usepackage{amssymb}
\usepackage{amsthm}

\usepackage{graphicx}

\usepackage{verbatim}

\usepackage{hyperref}


%no paragraph indent
\setlength{\parindent}{0pt}


\begin{document}

\begin{flushright}
Russ Johnson\\
Final Exam\\
\today\\
\end{flushright}

2.\\
(a) The operation table for $U_{44}$ is given in Table 1. Explain why $U_44=\langle [3] \rangle \times \langle [21] \rangle$, the internal direct product of the subgroups $\langle [3] \rangle$ and $\langle [21] \rangle$.\\
~\\
First of all, $\langle [3] \rangle = \{[1],[3],[9],[15],[23],[25],[27],[31],[37]\}$ and $\langle [21] \rangle = \{[1],[21]\}$. 
From this we see that the intersection of $\langle [3] \rangle$ and $\langle [21] \rangle$ contains only $[1]$. 
From Theorem 26.6 (2) we can conclude that each element in $\langle [3] \rangle \times \langle [21] \rangle$ has a unique representation $kn$ 
where $k\in \langle [3] \rangle$ and $n\in \langle [21] \rangle$. And so, 
$ |\langle [3] \rangle \times \langle [21] \rangle| = |\langle [3] \rangle|\cdot |\langle [21] \rangle | = 10\cdot 2 = 20. $ 
From the closure property of the group $G$ and the fact that $\langle [3] \rangle $ and $ \langle [21] \rangle$ are subgroups of $G$ we know that each element in $ \langle [3] \rangle \times \langle [21] \rangle $ is also in $G$. From this fact and the fact that $ \langle [3] \rangle \times \langle [21] \rangle $ and $G$ have the same order, it must be the case that $ G= \langle [3] \rangle \times \langle [21] \rangle $\\
~\\
(b) 

\end{document}



References:
http://math.stackexchange.com/questions/14282/why-do-we-define-quotient-groups-for-normal-subgroups-only
http://www.proofwiki.org/wiki/Third_Isomorphism_Theorem/Groups
http://www.proofwiki.org/wiki/Coset_Product_is_Well-Defined

http://drexel28.wordpress.com/2011/01/02/review-of-group-theory-the-first-isomorphism-theorem/
http://drexel28.wordpress.com/2011/01/02/review-of-group-theory-the-third-isomorphism-theorem/

http://www.proofwiki.org/wiki/Equivalence_of_Definitions_of_Normal_Subgroup
