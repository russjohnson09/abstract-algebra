%Written using Texmaker.
%On linux 
%sudo apt-get install texlive-full
%sudo apt-get install texmaker
%Getting started
%http://en.wikibooks.org/wiki/LaTeX
\documentclass[11pt]{article}
\usepackage[a4paper]{geometry} 
\geometry{a4paper, top=1.2cm, left=1.5cm, right=2.0cm, bottom=.1cm, includehead, includefoot}
\usepackage{amsmath}
\usepackage{amsfonts}
\usepackage{amssymb}
\usepackage{amsthm}

\usepackage{graphicx}

\usepackage{verbatim}

\usepackage{hyperref}


%no paragraph indent
\setlength{\parindent}{0pt}


\begin{document}

\begin{flushright}
Russ Johnson\\
Final Exam\\
\today\\
\end{flushright}

2.\\
(a) The operation table for $U_{44}$ is given in Table 1. Explain why $U_44=\langle [3] \rangle \times \langle [21] \rangle$, the internal direct product of the subgroups $\langle [3] \rangle$ and $\langle [21] \rangle$.\\
~\\
First of all, $\langle [3] \rangle = \{[1],[3],[9],[15],[23],[25],[27],[31],[37]\}$ and $\langle [21] \rangle = \{[1],[21]\}$. 
From this we see that the intersection of $\langle [3] \rangle$ and $\langle [21] \rangle$ contains only $[1]$. 
From Theorem 26.6 (2) we can conclude that each element in $\langle [3] \rangle \times \langle [21] \rangle$ has a unique representation $kn$ 
where $k\in \langle [3] \rangle$ and $n\in \langle [21] \rangle$. And so, 
$ |\langle [3] \rangle \times \langle [21] \rangle| = |\langle [3] \rangle|\cdot |\langle [21] \rangle | = 10\cdot 2 = 20. $ 
From the closure property of the group $G$ and the fact that $\langle [3] \rangle $ and $ \langle [21] \rangle$ are subgroups of $G$ we know that each element in $ \langle [3] \rangle \times \langle [21] \rangle $ is also in $G$. From this fact and the fact that $ \langle [3] \rangle \times \langle [21] \rangle $ and $G$ have the same order, it must be the case that $ G= \langle [3] \rangle \times \langle [21] \rangle $\\
~\\
(b) When we decompose a group as an internal direct product, it is convenient for classification purposes to identify that internal direct product with an external direct product. Let $G$ be an arbitrary group with identity element $e$ and let $K$ and $N$ be normal subgroups of $G$ with $K \cap N = \{e\}$. Prove that 
\[ K \times N \cong (K \oplus N).\]\\
~\\
\newcommand{\dom}{\ensuremath{K \oplus N}}
\newcommand{\img}{\ensuremath{K \times N}}
\begin{proof}
We will prove that $\phi: K \oplus N \rightarrow K \times N$ such that $\phi((k,n)) = kn$ for all $k\in K$ and for all $n\in N$ is an isomorphism. In doing so we will have proven that $\dom$ and $\img$ are isomorphic and therefore $ (\dom) \cong (\img) $.\\
~\\
First we will show that the $\phi$ is in fact a function. This means that it is well-defined. Let $(k,n) = (k',n') \in \dom$. We will prove that $\phi((k,n)) = \phi((k',n'))$. We see that
\[ \phi((k,n)) = kn = k'n' \phi((k',n')) \]
and therefore $\phi$ is a function.\\
~\\
Next we will show that $\phi$ preserves structure and therefore is a homomorphism. Let $(k,n),(k',n')\in \dom$. First we see that 
\[ \phi((k,n)(k',n')) = \phi((kk',nn')) = (kk')(nn'). \]
From Theorem 26.6 (1) we know that $k'n=nk'$. Therefore,
\[ \phi((k,n)(k',n')) = (kk')(nn') = k(k'n)n' = k(nk')n' = (kn)(k'n') = \phi((k,n))\phi((k'n')) \]
and $\phi$ preserves the operation in $G$.\\
~\\
Finally, we will show that $\phi$ is both injective and surjective. Let $\phi((k,n)) = \phi((k',n'))$. First we note that
\[ kn = \phi((k,n)) = \phi((k',n')) = k'n' \]
and so $kn=k'n'$. From Theorem 26.6 (2) we know that $kn$ is a unique representation of an element in $\img$, which means that $k$ must be equal to $k'$ and $n$ must be equal to $n'$. Therefore, $(k,n) = (k',n')$ and $\phi$ is injective.\\
~\\
Let $kn\in \img$. The element $(k,n)\in \dom$ is such that $\phi((k,n))=kn$ and so $\phi is surjective$.
\end{proof}
~\\
Explain why $U_{44} \cong (\mathbb{Z}_{10} \oplus \mathbb{Z}_2)$.\\
~\\
Because $\mathbb{Z}_{10} \cap \mathbb{Z}_2 = [1]$, it follows that
 $\mathbb{Z}_{10} \times \mathbb{Z}_2 = U_{44}$ in a similar manner to part (a). Also from part (b) we know that $U_{44} = \mathbb{Z}_{10} \times \mathbb{Z}_2 \cong \mathbb{Z}_{10} \oplus \mathbb{Z}_2$. 
Therefore, $U_{44} \cong (\mathbb{Z}_{10} \oplus \mathbb{Z}_2)$.\\
~\\
3.\\
(a) Let $G= U_{44}$. Let $N = \langle [3] \rangle$ and let $K=\langle [9] \rangle$. The operation table for $U_{44}$ is shown in Table 1. Find $G/K$, $G/N$, $N/K$, and $(G/K)/(N/K)$. Explain why $(G/K)/(N/K) \cong G/N$.
First of all,
\[ G/K = \{K, [3]K, [7]K, [13]K\},\]
\[ G/N = \{N, [7]N\}, \]
\[ N/K = \{K, [3]K\},\]
and
\[(G/K)/(N/K) = \{ N/K, ([7]K)(N/K) \}.\]
The groups $(G/K)/(N/K)$ and $G/N$ both have only two elements and therefore are both isomorphic with $\mathbb{Z}_2$. And so, $(G/K)/(N/K) \cong G/N$.

\end{document}



References:
http://math.stackexchange.com/questions/14282/why-do-we-define-quotient-groups-for-normal-subgroups-only
http://www.proofwiki.org/wiki/Third_Isomorphism_Theorem/Groups
http://www.proofwiki.org/wiki/Coset_Product_is_Well-Defined

http://drexel28.wordpress.com/2011/01/02/review-of-group-theory-the-first-isomorphism-theorem/
http://drexel28.wordpress.com/2011/01/02/review-of-group-theory-the-third-isomorphism-theorem/

http://www.proofwiki.org/wiki/Equivalence_of_Definitions_of_Normal_Subgroup
