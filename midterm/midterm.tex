\documentclass[11pt,a4paper]{article}

\usepackage{amsmath}
\usepackage{amsfonts}
\usepackage{amssymb}
\usepackage{amsthm}

\usepackage{graphicx}

\usepackage{verbatim}

\usepackage{hyperref}


%no paragraph indent
\setlength{\parindent}{0pt}


\begin{document}

\begin{flushright}
Russ Johnson\\
Midterm Exam\\
\today\\
\end{flushright}

{\bf 1.}\\
Let $H$ be a non-empty subset of a group $G$ with identity $e$.\\
{\bf Conjecture.} $H$ is a subgroup of $G$ if and only if whenever $h$ and $k$ are in $H$, then $h^{-1}k$ is in $H$.\\
~\\
We will first prove that $H$ is a subgroup of $G$ with identity $e$ if whenever $h$ and $k$ are in $H$, then $h^{-1}k$ is in $H$.
\begin{proof}
Let $a\in H$. Because $a\in H$ we now that $a^{-1}a$ is also in $H$ and coming from the fact that $a^{-1}a = e$, we know that $e\in H$.\\
~\\
Now that we now that $H$ has an identity element, it becomes easy to prove that every element in $H$ has an inverse. Again, let $a\in H$. We know that $e\in H$ and so $a^{-1}e = a^{-1} \in H$, and so every element in $H$ has an inverse.\\
~\\
Finally, let $a,b \in H$. We also now that $a^{-1}\in H$, coming from the fact that every element in $H$ has an inverse. Therefore, $(a^{-1})^{-1}b = ab\in H$ and from this we see that $H$ is closed under the operation in group $G$.
\end{proof}

Next we will prove that if $H$ is a subgroup $G$, then whenever $h$ and $k$ are in $H$, $h^{-1}k$ is in $H$.

\begin{proof}
Let $h,k\in H$. We know that $h^{-1}$ is in $H$, because $H$ is a group and therefore every element in $H$ has an inverse. We also know from the fact that $H$ is a group it must be closed under its operation. Therefore, $h^{-1}k \in H$.
\end{proof} 

From these two proofs we can conclude that $H$ is a subgroup of $G$ if and only if whenever $h$ and $k$ are in $H$, then $h^{-1}k$ is in $H$.\\

{\bf 2.}\\
{\bf (a)}\\
The set $HK$ is equal to $\{[1],[2],[4],[8],[11],[16]\}$.\\
~\\
{\bf (b)}\\
Let $G$ be an Abelian group with identity $e$ and subgroups $H$, and $K$. Let 
\[HK = \{hk:h\in H \land k\in K\}.\]
{\bf Conjecture.} The set $HK$ is subgroup of $G$.
\begin{proof}
Let $H$ and $K$ be subgroups of $G$ with identity $e$. We know that $e\in H$ and $e\in K$ from the fact that $H$ and $K$ are subgroups of $G$. We know from the definition of the set $HK$ that $ee = e\in HK$. And so, $HK$ contains the identity of $G$.\\
~\\
Let $a,b \in HK$. From the definition of $HK$ we know that there exists some elements $h_1,h_2\in H$ and $k_1,k_2\in K$ such that $ab = (h_1k_1)(h_2k_2)$. We know that the operation of $G$ is commutative and so be applying this rule and the associative property of the group $G$ we know that 
\[ab = (h_1k_1)(h_2k_2) = h_1(k_1h_2)k_2 = h_1(h_2k_1)k_2 = (h_1h_2)(k_1k_2). \]
The groups $H$ and $K$ are closed under the operation in $G$ and so $h_1h_2\in H$ and $k_1k_2 \in K$. From the definition of $HK$ we know that $ab =(h_1h_2)(k_1k_2)\in HK $. In conclusion, $HK$ is closed under the operation of $G$.\\
~\\
Finally, let $a\in HK$. From the definition of $HK$ we know that there exists some elements $h\in H$ and $k\in K$ such that $a = hk$. We know that $h^{-1}\in H$ and $k^{-1}\in H$ coming from the fact that $H$ and $K$ are subgroups of $G$. From this, the fact that the operation of $G$ is commutative, and the associative property of the group $G$ we know that
\[(k^{-1}h^{-1})a = a(k^{-1}h^{-1}) = (hk)(k^{-1}h^{-1}) = h(kk^{-1})h^{-1} = (he)h^{-1} = hh^{-1} = e. \]
And so, the inverse of $a$ is $k^{-1}h^{-1}$ and from the definition of $HK$ we know that $k^{-1}h^{-1} \in HK$. In conclusion, we have shown that every element in $HK$ has an inverse.
\end{proof}
By proving that $HK$ contains the identity element $e\in G$, the set $HK$ is closed under the operation of $G$, and ever element in $HK$ has an inverse, we have proven that $HK$ is a subgroup of $G$.\\
~\\
{\bf (c)}\\
The set $HK$ is not necessarily a subgroup of $G$ if $G$ is non-Abelian. In our proof we relied on the fact that $G$ is Abelian to show that $HK$ is closed under the operation of $G$. We will provide a counter-example with the group $G$ with identity $e$, operator $\cdot$, the operation table
\[
\begin{array}{c|c|c|c|c}
\cdot & e & a & b & c \\ \hline
e & e & a & b & c \\ \hline
a & a & e & b & b \\ \hline
b & b & c & e & a \\ \hline
c & c & a & a & e
\end{array} 
\]
and the subgroups $H$ and $K$ with operation tables 
\[
\begin{array}{c|c|c}
\cdot & e & a \\ \hline
e & e & a \\ \hline
a & a & e
\end{array} 
\]
and
\[
\begin{array}{c|c|c}
\cdot & e & b \\ \hline
e & e & b \\ \hline
b & b & e
\end{array} 
\]
respectively.\\
We see that $HK = \{e,a,b\}$ and $ba = c$. Therefore, $HK$ is not closed under the operation $\cdot$ and is not a subgroup of $G$.\\
~\\
{\bf 3.}\\
The first symmetry is the identity, \[\begin{pmatrix}1&2&3&4\\1&2&3&4\end{pmatrix}.\]
The next set of symmetries are rotations. The first of these three is a $180$ degree rotation about the axis through the midpoints of edges $12$ and $34$. In permutation notation this is \[\begin{pmatrix}1&2&3&4\\2&1&4&3\end{pmatrix} = (12)(34).\] The next two are all $180$ degree rotations about some axis. The first is through the midpoints of edges $13$ and $24$ and the second is through the midpoints of edges $14$ and $23$. In permutation notation these are 
\[\begin{pmatrix}1&2&3&4\\3&4&1&2\end{pmatrix} = (13)(24)\] 
and
\[\begin{pmatrix}1&2&3&4\\4&3&2&1\end{pmatrix} = (14)(23)\] 
respectively.\\
The next set of symmetries are rotations. The first of these four is a $120$ degree rotation counterclockwise about the axis through the vertex $1$ and the center of $\triangle 234$. In permutation notation this is \[\begin{pmatrix}1&2&3&4\\1&3&4&2\end{pmatrix} = (234).\] The next three are all $120$ degree rotations about some axis. The first is through vertex $2$ and the center of $\triangle 134$, the second is through vertex $3$ and the center of $\triangle 124$, and the last is through vertex $4$ and the center of $\triangle 123$. In permutation notation these are
\[\begin{pmatrix}1&2&3&4\\4&2&1&3\end{pmatrix} = (143),\]
\[\begin{pmatrix}1&2&3&4\\2&4&3&1\end{pmatrix} = (124),\]
and
\[\begin{pmatrix}1&2&3&4\\3&1&2&4\end{pmatrix} = (132)\]
respectively.\\
The last set of symmetries are rotations. The first of these four is a $240$ degree rotation counterclockwise about the axis through the vertex $1$ and the center of $\triangle 234$. In permutation notation this is
\[\begin{pmatrix}1&2&3&4\\1&4&2&3\end{pmatrix} = (243).\]
The next three are all $240$ degree rotations about some axis. The first is through vertex $2$ and the center of $\triangle 134$, the second is through vertex $3$ and the center of $\triangle 124$, and the last is through vertex $4$ and the center of $\triangle 123$. In permutation notation these are
\[\begin{pmatrix}1&2&3&4\\3&2&4&1\end{pmatrix} = (134),\]
\[\begin{pmatrix}1&2&3&4\\4&1&3&2\end{pmatrix} = (142),\]
and
\[\begin{pmatrix}1&2&3&4\\2&3&1&4\end{pmatrix} = (123)\]
respectively.\\
All together we have the symmetries
\[I = \begin{pmatrix}1&2&3&4\\1&2&3&4\end{pmatrix},\]
\[R_{12} = \begin{pmatrix}1&2&3&4\\2&1&4&3\end{pmatrix} = (12)(34),\]
\[R_{13} = \begin{pmatrix}1&2&3&4\\3&4&1&2\end{pmatrix} = (13)(24),\]
\[R_{14} = \begin{pmatrix}1&2&3&4\\4&3&2&1\end{pmatrix} = (14)(23),\]
\[R_1 = \begin{pmatrix}1&2&3&4\\1&3&4&2\end{pmatrix} = (234),\]
\[R_2 = \begin{pmatrix}1&2&3&4\\4&2&1&3\end{pmatrix} = (143),\]
\[R_3 = \begin{pmatrix}1&2&3&4\\2&4&3&1\end{pmatrix} = (124),\]
\[R_4 = \begin{pmatrix}1&2&3&4\\3&1&2&4\end{pmatrix} = (132),\]
\[R_1^2 = \begin{pmatrix}1&2&3&4\\1&4&2&3\end{pmatrix} = (243).\]
\[R_2^2 = \begin{pmatrix}1&2&3&4\\3&2&4&1\end{pmatrix} = (134),\]
\[R_3^2 = \begin{pmatrix}1&2&3&4\\4&1&3&2\end{pmatrix} = (142),\]
and
\[R_4^2 = \begin{pmatrix}1&2&3&4\\2&3&1&4\end{pmatrix} = (123).\]
The operation table for this group of symmetries is

\[
\begin{array}{c|c|c|c|c|c|c|c|c|c|c|c|c}
\circ &I &R_{12} &R_{13} &R_{14} &R_1 &R_2 &R_3 &R_4 &R_1^2 &R_2^2 &R_3^2 &R_4^2 \\ \hline
 I & I & R_{12} & R_{13} & R_{14} & R_1 & R_2 & R_3 & R_4 & R_1^2 & R_2^2 & R_3^2 & R_4^2 \\\hline
 R_{12} & R_{12} & I & R_{14} & R_{13} & R_3 & R_4 & R_1 & R_2 & R_4^2 & R_3^2 & R_2^2 & R_1^2\\\hline
 R_{13} & R_{13} & R_{14} & I & R_{12} & R_4 & R_3 & R_2 & R_1 & R_2^2 & R_1^2 & R_4^2 & R_3^2 \\\hline
 R_{14} & R_{14} & R_{13} & R_{12} & I & R_2 & R_1 & R_4 & R_3 & R_3^2 & R_4^2 & R_1^2 & R_2^2 \\\hline
 R_1 & R_1 & R_4 & R_2 & R_3 & R_1^2 & R_4^2 & R_2^2 & R_3^2 & I & R_{14} & R_{12} & R_{13} \\\hline
 R_2 & R_2 & R_3 & R_1 & R_4 & R_3^2 & R_2^2 & R_4^2 & R_1^2 & R_{14} & I & R_{13} & R_{12} \\\hline
 R_3 & R_3 & R_2 & R_4 & R_1 & R_4^2 & R_1^2 & R_3^2 & R_2^2 & R_{12} & R_{13} & I & R_{14} \\\hline
 R_4 & R_4 & R_1 & R_3 & R_2 & R_2^2 & R_3^2 & R_1^2 & R_4^2 & R_{13} & R_{12} & R_{14} & I \\\hline
 R_1^2 & R_1^2 & R_3^2 & R_4^2 & R_2^2 & I & R_{13} & R_{14} & R_{12} & R_1 & R_3 & R_4 & R_2 \\\hline
 R_2^2 & R_2^2 & R_4^2 & R_3^2 & R_1^2 & R_{13} & I & R_{12} & R_{14} & R_4 & R_2 & R_1 & R_3 \\\hline
 R_3^2 & R_3^2 & R_1^2 & R_2^2 & R_4^2 & R_{14} & R_{12} & I & R_{13} & R_2 & R_4 & R_3 & R_1 \\
\end{array}
.\]




\end{document}

\[
\begin{array}{c|c|c|c|c|c|c|c|c|c|c|c|c}
\circ & I & R_{12} & R_{13} & R_{14} & R_1 & R_2 & R_3 & R_4 & R_1^2 & R_2^2 & R_3^2 & R_4^2 \\ \hline
I & I & R_{12} & R_{13} & R_{14} & R_1 & R_2 & R_3 & R_4 & R_1^2 & R_2^2 & R_3^2 & R_4^2\\ \hline
R_{12} & R_{12} & I & R_{14} & R_{13} & R_4 & R_3 & R_2 & R_1 & R_3^2 & R_4^2 & R_1^2 & R_2^2 \\ \hline
R_{13} & R_{13} & R_{14} & I & R_{12} & R_2 & R_1 & R_4 & R_3 & R_4^2 & R_3^2 & R_2^2 & R_1^2 \\ \hline
R_{14} & R_{14} & R_{13} & R_{12} & I & R_3 & R_4 & R_1 & R_2 & R_2^2 & R_1^2 & R_4^2 & R_3^2 \\ \hline
R_1 & R_1 & R_3 & R_4 & R_2 & R_1^2 & R_3^2 & R_4^2 & R_2^2 & I & R_{13} & R_{14} & R_{12} \\ \hline
R_2 & R_2 & R_4 & R_3 & R_1 & R_4^2 & R_2^2 & R_1^2 & R_3^2 & R_{13} & I & R_{12} & R_{14} \\ \hline
R_3 & R_3 & R_1 & R_2 & R_4 & R_2^2 & R_4^2 & R_3^2 & R_1^2 & R_{14} & R_{12} & I & R_{13} \\ \hline
R_4 & R_4 & R_2 & R_1 & R_3 & R_3^2 & R_1^2 & R_2^2 & R_4^2 & R_{12} & R_{14} & R_{13} & I \\ \hline
R_1^2 & R_1^2 & R_4^2 & R_2^2 & R_3^2 & I & R_{14} & R_{12} & R_{13} & R_1 & R_4 & R_2 & R_3 \\ \hline
R_2^2 & R_2^2 & R_3^2 & R_1^2 & R_4^2 & R_{14} & I & R_{13} & R_{12} & R_3 & R_2 & R_4 & R_1 \\ \hline
R_3^2 & R_3^2 & R_2^2 & R_4^2 & R_1^2 & R_{12} & R_{13} & I & R_{14} & R_4 & R_1 & R_3 & R_2 \\ \hline
R_4^2 & R_4^2 & R_1^2 & R_3^2 & R_2^2 & R_{13} & R_{12} & R_{14} & I & R_2 & R_3 & R_1 & R_4
\end{array} 
.\]

The next set of symmetries are reflections. The first is a reflection through the plane containing edge $34$ and the midpoint of $12$. In permutation notation this is \[\begin{pmatrix}1&2&3&4\\2&1&3&4\end{pmatrix} = (12).\] The next five are the reflections through the plane containing edge $24$ and the midpoint of $13$, the plane containing edge $23$ and the midpoint of $14$, the plane containing the edge $14$ and the midpoint of $23$, the plane containing the edge $13$ and the midpoint of $24$, and the plane containing the edge $12$ and the midpoint of $34$. In permutation notation these are \[\begin{pmatrix}1&2&3&4\\3&2&1&4\end{pmatrix} = (13),\] \[\begin{pmatrix}1&2&3&4\\4&2&3&1\end{pmatrix} = (14),\] \[\begin{pmatrix}1&3&2&4\\1&2&3&4\end{pmatrix} = (23),\] \[\begin{pmatrix}1&2&3&4\\1&4&3&2\end{pmatrix} = (24),\] and \[\begin{pmatrix}1&2&3&4\\1&2&4&3\end{pmatrix} = (34)\]
respectively.\\