\documentclass[11pt,a4paper]{article}

\usepackage{amsmath}
\usepackage{amsfonts}
\usepackage{amssymb}
\usepackage{amsthm}

\usepackage{graphicx}

\usepackage{verbatim}

\usepackage{hyperref}


%no paragraph indent
\setlength{\parindent}{0pt}


\begin{document}

\begin{flushright}
Russ Johnson\\
Problem Set $\#5$\\
\today\\
\end{flushright}

Activity 24.12 Write a formal proof of Lagrange's Theorem (Theorem 24.4).
\begin{proof}
Let $G$ be a finite group and let $H$ be a subgroup of $G$. From Theorem 24.3 we know that the left cosets of $H$ form a partition of $G$. Let $n$ be the number of distinct left cosets of $H$ in $G$. Each left coset of $H$ contains $|H|$ elements (Activity 24.7). And so, $|G| = n|H|$. Therefore, from the definition of divides we can conclude that $|H|$ divides $|G|$. In conclusion, if $H$ is a subgroup of a finite group $G$, then $|H|||G|$.
\end{proof}
Activity 24.13. Let $G$ be a group with identity $e$ and assume that $\sim$ is a congruence relation on $G$.\\
(a) Let $H = \{x \in G : x \sim e\}$. Show that $H$ is a subgroup of $G$.\\
~\\
First of all, we know that $e\sim e$ from the reflexive property of the congruence relation $\sim$. And so, the identity element of $G$ is also in $H$.\\
~\\
Next, let $a\in H$. From the definition of the set $H$, we know that $a\sim e$. Because $\sim$ is a congruence relation,  $a^{-1} \sim e^{-1} $. And so, $a^{-1} \sim e$ which means that $a^{-1}\in H$. In conclusion, every element in $H$ has an inverse that is also in $H$. (Which is the same as the inverse of this element in $G$).\\
~\\
Finally, let $a,b\in H$. From the definition of the set $H$, we know that $a\sim e$ and $b\sim e$. Because $\sim$ is a congruence relation,  $ab \sim e^2 $. And so $ab \sim e$ which means that $ab\in H$ from the definition of the set $H$. Therefore, the operation of $G$ is closed in $H$.\\
~\\
In conclusion, the identity of $G$ is in $H$, every element in $H$ has an inverse, and the operator of $G$ is closed in the set $H$. Therefore, the set $H$ is a subgroup of $G$.\\
~\\
(b) Let $a, b \in G$. Prove that $a \sim b$ if and only if $a^{-1} b \in H$.\\
\begin{proof}
First, let $a,b\in G$ such that 
\begin{equation}\label{b1}
a \sim b.
\end{equation}
From the reflexive property of the congruence relation $\sim$ we know that 
\begin{equation}\label{b2}
a ^{-1}\sim a^{-1}.
\end{equation}
From part 1 of the definition of a congruence relation we can take \eqref{b1} and \eqref{b2} to obtain
\begin{equation}\label{b3}
aa^{-1} \sim ab^{-1}.
\end{equation}
From \eqref{b3} we know that $e \sim ab^{-1}$ and from the symmetric property we know that $ab^{-1} \sim e$. From the definition of the set $H$, $ab^{-1}\in H$.\\
~\\
Next, let $a,b\in G$ such that $a^{-1}b \in H$ which from its definition means
\begin{equation}\label{b4}
a^{-1}b \sim e.
\end{equation}
From the reflexive property of the congruence relation we know that 
\begin{equation}\label{b5}
a \sim a.
\end{equation}
From part 1 of the definition of a congruence relation we know from \eqref{b4} and \eqref{b5} that
\begin{equation}\label{b6}
aa^{-1}b \sim ae.
\end{equation}
Simplifying \eqref{b6} we arrive at
\begin{equation}\label{b7}
b \sim a.
\end{equation}
Applying the symmetric property of the congruence relation $\sim$ to \eqref{b7} we obtain
$a \sim b$. From this we have show that if $a\sim b$, then $ab^{-1} \in H$ and if $ab^{-1} \in H$, then $a\sim b$. In other words, $a\sim b$ if and only if $ab^{-1} \in H$.
\end{proof}

(c) Explain why $\sim_H$ is the only possible congruence relation on a group $G$.\\
~\\
At the beginning of this activity the only assumption that we made was that $\sim$ is a congruence relation on $G$. In parts (a) and (b), we proved that the congruence relation $\sim$ is equivalent to $\sim_H$. This means that if $G$ has a congruence relation, then this congruence relation is $\sim_H$. In other words, $\sim_H$ is the only possible congruence relation on $G$.\\
~\\
Activity 24.14.\\
(a) State the converse of Lagrange's Theorem. What do we need to do to show that the converse of Lagrange's Theorem is not true?\\
~\\
{\bf Converse.} If  $m$ is a divisor of a finite group $G$, then there exists some subgroup of $G$ with order $m$.\\
~\\
To prove that the converse of Lagrange's Theorem is false we need to give an example of group that has a divisor with no possible subgroup of that order.\\
~\\
(b) Consider the group $G = A_4$ . List the elements of $A_4$ in cycle notation and determine the order of $A_4$.\\
~\\
The order of $A_4$ is $12$ and 
\[A_4 = \{ I, (12)(34), (13)(24), (14)(23), (123), (132), (124), (142), (134), (143), (234), (243) \}.\]
\\
(c) Assume that $H$ is a subgroup of $A_4$ of order $6$.\\
(i) Explain why the nonidentity elements of $H$ must have order $2$ or $3$.\\
~\\
The order of the elements of a subgroup are equal to the order of these same elements in their parent group due to the closure property of subgroups. The nonidentity elements $(12)(34)$, $(13)(24)$, and $(14)(23)$ in $A_4$ have an order of $2$. The other non-identity all have an order of $3$. Therefore the order of the nonidentity elements in $H$ must have order $2$ or $3$.\\
~\\
(ii) Explain why there must be an element $\alpha$ of $A_4$ of order $3$ that is not in $H$.\\
~\\
The elements $(123)$, $(142)$, and $(134)$ all have order $3$. Because $H$ is a subgroup of $A_4$, it must be closed and in order to have all three of $(123)$, $(142)$, and $(134)$ $H$ would need to have an order of at least $9$.\\
~\\
(iii) Explain why the left cosets $H$, $\alpha H$ and $\alpha^2 H$ cannot all be distinct.\\
~\\
From Theorem 24.3 part 2 we know that the group $G$ can be written as a disjoint union of left cosets.
If we assume that they are all distinct, from Theorem 24.3 part 1 we know that they cannot share any elements. The total number of elements for each of these sets is six and so $A_4$ must contain at least $18$ elements in order for all of the left cosets to be distinct. Because this is not the case, at least two of these left cosets must be equal.\\
~\\
(iv) Show that it is not possible for any two of $H$, $\alpha H$ and $\alpha^2 H$ to be equal.\\
~\\
We know that $\alpha$ has order $3$ and so $e$, $\alpha$ and $\alpha^2$ are all unique elements. Because $H$ is a subgroup of $G$ we know that $G$'s identity, $e$, is in $H$. We also see that $\alpha \in \alpha H$ and $\alpha^2 \in \alpha^2 H$. Because $\alpha$ is not in $H$, $\alpha \notin H $, $\alpha^2 \notin \alpha H$, and $\alpha^3 = e \notin \alpha^2 H$. From this we see that $H \neq \alpha H$, $\alpha H \neq \alpha^2 H$, and $\alpha^2 H \neq H$.\\
~\\
(d) Explain why the converse of Lagrange's Theorem is not true.\\
~\\
The group $A_4$ does not have a subgroup of order $6$ and so the converse is not true. When we tried to construct a subgroup of $A_4$ with order six we arrived at a contradiction. We had the three left cosets $H$, $\alpha H$ and $\alpha^2 H$ that could not all be distinct, but none could be equal.\\
~\\
(4) A group $G$ contains elements of every order from $1$ to $10$. What is the smallest order $G$ could have? Find a group $G$ of that order that contains elements of every order from $1$ through $10$.\\
~\\
By Lagrange's Theorem the order of a subgroup must divide the order of the group. The order of an element is equal to the number of elements in the subgroup generated by that element. Therefore, a group with elements of every order from $1$ through $10$ must be divisible by the numbers $1$ through $10$. The least common multiple of the numbers $1$ through $10$ is the lowest of these numbers.
Therefore, the smallest order that $G$ could be is $lcm(1,2,3,4,5,6,7,8,9,10) = 2^3\cdot 3^2\cdot 5 \cdot 7 = 2520$. An example of a group with this order is $\mathbb{Z}_{2520}$ under addition. The identity, $[2520/1] = [0]$, has order $1$, $[2520/2] = [1260]$ has order $2$,  $[2520/3] = [840]$ has order $3$, and so on. Because $2520$ is divisible by each of the numbers $1$ through $10$, we are able to follow this pattern to obtain elements with the orders $1$ through $10$.\\
~\\
(6) Let $H = \{I, r\}$ in $D_4$.\\
(a) Determine all of the distinct left cosets of $H$ in $D_4$.\\
First of all, $D_4 = \{I, R, R^2, R^3, r, Rr, R^2r, R^3r\}$ and so $|D_4| = 8$. The number of distinct left cosets of $H$ in $D_4$ is referred to as the index and is $\dfrac{|D_4|}{|H|} = 4$. These four distinct left cosets are
\[IH = rH = \{I, r\},\] 
\[RH = RrH = \{R, Rr\},\]
\[R^2H = R^2rH = \{R^2, R^2r\},\]
and
\[R^3H = R^3rH = \{R^3, R^3r\}.\]
(b) Determine all of the distinct right cosets of $H$ in $D_4$.\\
Like in part (a), there are four distinct right cosets. These right cosets are
\[HI = Hr = \{I,r\},\]
\[HR = HR^3r = \{R,R^3\},\]
\[HR^2 = HR^2r = \{R^2, R^2r\},\]
and
\[HR^3 = HRr = \{R^3,Rr\}.\]
\end{document}