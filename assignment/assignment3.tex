\documentclass[11pt,a4paper]{article}

\usepackage{amsmath}
\usepackage{amsfonts}
\usepackage{amssymb}
\usepackage{amsthm}

\usepackage{graphicx}

\usepackage{verbatim}

\usepackage{hyperref}


%no paragraph indent
\setlength{\parindent}{0pt}


\begin{document}

\begin{flushright}
Russ Johnson\\
Problem Set $\#3$\\
\today\\
\end{flushright}

{\bf (15)} Let $\mathcal{F}(\mathbb{R})$ denote the set of all functions from $\mathbb{R}$ to $\mathbb{R}$. Define addition and multiplication on $\mathcal{F}(\mathbb{R})$ as follows:
\begin{itemize}
\item For all $f,g \in \mathcal{F}(\mathbb{R}), (f + g) :\mathbb{R} \rightarrow \mathbb{R}$ is the function defined by
\[(f+g)(x) = f(x)+g(x)\]
for all $x\in\mathbb{R}$.
\item $f,g \in \mathcal{F}(\mathbb{R}), (fg) :\mathbb{R} \rightarrow \mathbb{R}$ is the function defined by
\[(fg)(x) = f(x)g(x)\]
for all $x\in\mathbb{R}$.
\end{itemize}
~\\
(a) Prove that $\mathcal{F}(\mathbb{R})$ is an Abelian group under addition.\\
~\\
For this proof we will first show that $\mathcal{F}(\mathbb{R})$ is closed under addition, that addition is associative in $\mathcal{F}(\mathbb{R})$, that $\mathcal{F}(\mathbb{R})$ contains an identity element, and that each element in $\mathcal{F}(\mathbb{R})$ has an inverse. Finally, we will show that addition is commutative in $\mathcal{F}(\mathbb{R})$.\\
~\\
First of all, from the definition of addition in $\mathcal{F}(\mathbb{R})$ we see that this operation will always give us another function in $\mathcal{F}(\mathbb{R})$. Therefore, $\mathcal{F}(\mathbb{R})$ is closed under addition. Next we will prove that  addition is associative in $\mathcal{F}(\mathbb{R})$.
\begin{proof}
Let $f,g,h\in\mathcal{F}(\mathbb{R})$ and let $x\in\mathbb{R}$. From the definition of addition in $\mathcal{F}(\mathbb{R})$ we know that
\begin{equation}\label{151}
((f+g)+h)(x) = (f(x)+g(x))+h(x)
\end{equation}
and
\begin{equation}\label{152}
(f+(g+h))(x) = f(x)+(g(x)+h(x))
\end{equation}
We also know that
\begin{equation}\label{153}
f(x) \in \mathbb{R},
\end{equation}
\begin{equation}\label{154}
g(x) \in \mathbb{R},
\end{equation}
and
\begin{equation}\label{155}
h(x) \in \mathbb{R}
\end{equation}
from the fact that the codomain of $f,g,$ and $h$ is $\mathbb{R}$.
We already know that the addition is associative in $\mathbb{R}$. From this fact and the equations \eqref{153}, \eqref{154}, and \eqref{155}. We know that
\begin{equation}\label{156}
(f(x)+g(x))+h(x) = f(x)+(g(x)+h(x)).
\end{equation}
From applying the transitive property of equality to \eqref{156} and \eqref{151}, we obtain
\begin{equation}\label{157}
((f+g)+h)(x) = f(x)+(g(x)+h(x)).
\end{equation}
Applying this same property again to \eqref{157} and \eqref{152}, we obtain
\begin{equation}\label{158}
((f+g)+h)(x) = (f+(g+h))(x).
\end{equation}
Finally, from \eqref{158} we see that addition is associative in $\mathcal{F}(\mathbb{R})$.
\end{proof}
Next we will prove that $\mathcal{F}(\mathbb{R})$ contains an identity element.
\begin{proof}
Let $e\in\mathcal{F}(\mathbb{R})$ such that 
\begin{equation}
e(x) = 0
\end{equation}
for all $x\in\mathbb{R}$. Let $f\in\mathcal{F}(\mathbb{R})$. We see that
\[(f+e)(x) = f(x)+e(x)=f(x)+0=f(x)\]
and
\[(e+f)(x) = e(x)+f(x)=0+f(x)=f(x).\]
In conclusion, we have shown that $(f+e)(x)=(e+f)(x)=f(x)$ for all $x\in\mathbb{R}$ and therefore $e$ is the identity element in $\mathcal{F}(\mathbb{R})$.
\end{proof}
Next will show that each function in $\mathcal{F}(\mathbb{R})$ has an inverse under addition.
\begin{proof}
Let $f\in\mathcal{F}(\mathbb{R})$ and let $x\in\mathbb{R}$. First we note that $(f-f)(x)=f(x)-f(x)=0$ coming from the fact that $f(x)\in\mathbb{R}$ and each element in $\mathbb{R}$ has an inverse under addition. We know that  for a We know that $\mathbb{R}$ is a group under addition. Therefore each element in $\mathbb{R}$ has an inverse
Let $f\in\mathcal{F}(\mathbb{R})$ and let $g\in\mathcal{F}(\mathbb{R})$ such that 
\end{proof}
Finally, we will show that every element in $\mathcal{F}(\mathbb{R})$ commutes under addition.
\begin{proof}
Let $f,g\in\mathcal{F}(\mathbb{R})$ and let $x\in\mathbb{R}$. We know that
\begin{equation}\label{15e1}
(f+g)(x)= f(x)+g(x)
\end{equation}
and
\begin{equation}\label{15e2}
(g+f)(x)=g(x)+f(x)
\end{equation}
from the definition of a function in $\mathcal{F}(\mathbb{R})$. We also know that
\begin{equation}\label{15e3}
f(x)\in\mathbb{R}
\end{equation}
and
\begin{equation}\label{15e4}
g(x)\in\mathbb{R}
\end{equation}
from the fact that a function in $\mathcal{F}(\mathbb{R})$ has the codomain $\mathbb{R}$. Because addition is commutative in $\mathbb{R}$, from \eqref{15e3} and \eqref{15e4} we can conclude that
\begin{equation}\label{15e5}
f(x)+g(x)=g(x)+f(x).
\end{equation}
Applying the transitive property of equality to \eqref{15e1} and \eqref{15e5} we obtain
\begin{equation}\label{15e6}
(f+g)(x)=g(x)+f(x).
\end{equation}
Applying this same property to \eqref{15e6} and \eqref{15e2} we obtain
\[(g+f)(x)=(f+g)(x).\]
In conclusion, we have shown that $(g+f)(x)=(f+g)(x)$ for all $x\in\mathbb{R}$ and therefore addition is commutative in $\mathcal{F}(\mathbb{R})$.
\end{proof}
~\\
In conclusion, we have shown that $\mathcal{F}(\mathbb{R})$ is closed under addition, addition is associative in $\mathcal{F}(\mathbb{R})$, $\mathcal{F}(\mathbb{R})$ contains an identity element, each element in $\mathcal{F}(\mathbb{R})$ has an inverse, and addition is commutative in $\mathcal{F}(\mathbb{R})$. From this we know that $\mathcal{F}(\mathbb{R})$ is an Abelian group under addition.\\
~\\
(b) Does $\mathcal{F}(\mathbb{R})$ have an identity element for multiplication?\\
~\\
Yes, let $e\in\mathcal{F}(\mathbb{R})$ such that
\[e(x)=1\]
for all $x\in\mathbb{R}$. Let $f\in\mathcal{F}(\mathbb{R})$ and let $a\in\mathbb{R}$. We see that
\[(fe)=f(a)e(a)=f(a)\cdot 1=f(a)\]
and
\[(ef)=e(a)f(a)=1\cdot f(a)=f(a).\]
Therefore $e$ is the identity in $\mathcal{F}(\mathbb{R})$ under multiplication.\\
~\\
(c) Find an element in $\mathcal{F}(\mathbb{R})$ that does not have a multiplicative inverse in $\mathcal{F}(\mathbb{R})$. Explain how this shows $\mathcal{F}(\mathbb{R})$ is not a group under multiplication.\\
~\\
Let $f\in\mathcal{F}(\mathbb{R})$ such that
\[f(x)=0\]
for all $x\in\mathbb{R}$. Let $g\in\mathcal{F}(\mathbb{R})$ and let $a\in\mathbb{R}$. We see that
\[(fg)(a)=0\cdot g(a)=0\neq 1.\]
And from this we can conclude that the function $f$ has no inverse.\\
~\\
(d) Find necessary and sufficient conditions for an element in $\mathcal{F}(\mathbb{R})$ to be a unit in $\mathcal{F}(\mathbb{R})$. State your result in a lemma of the form ``The function $f\in\mathcal{F}(\mathbb{R})$ is a unit in $\mathcal{F}(\mathbb{R})$ if and only if ...''. Your lemma must say something more than just a rehash of the definition of a unit; rather, it must actually characterize the functions that are invertible under multiplication in $\mathcal{F}(\mathbb{R})$.\\
~\\
{\bf Conjecture.} An element in $\mathcal{F}(\mathbb{R})$ is a unit if and only if $f(x)\neq 0$ for all $x\in\mathbb{R}$.\\
~\\
First we will show that an if $f$ is a function in $\mathcal{F}(\mathbb{R})$ such that $f(x)\neq 0$ for all $x\in\mathbb{R}$, then $f$ is a unit.
\begin{proof}
Let $f$ be a function in $\mathcal{F}(\mathbb{R})$ such that $f(x)\neq 0$ for all $x\in\mathbb{R}$. There exists some $g$ in $\mathcal{F}(\mathbb{R})$ such that $(fg)(x) = (gf)(x)= 1$ for all $x\in\mathbb{R}$. Let $a\in\mathbb{R}$. Let $g$ be a function in $\mathcal{F}(\mathbb{R})$ such that $f(a)g(a) = 1$.
\end{proof}
~\\
{\bf Activity 20.12.} In this activity, we will explore a simple relationship between the order of a group element and the order of its inverse.\\
(a) Determine the order of $[2]$ in $\mathbb{Z}_6$ . What is the inverse of $[2]$ in $\mathbb{Z}_6$ ? Directly
compute the order of the inverse of $[2]$ in $\mathbb{Z}_6$ . What do you notice?\\
~\\
First of all, we note that $\langle [2] \rangle = \{[0],[2],[4]\}$. The magnitude of this set is $3$ and therefore the order of $[2]$ in $\mathbb{Z}_6$ is $3$. The inverse of $[2]$ is $[4]$, ($[2]+[4]=[0]$). The order of $[4]$ in $\mathbb{Z}_6$ is equal to the magnitude of $\langle [4] \rangle = \{[0],[2],[4]\}$, and so the order of $[4]$ is $3$. The sets $\langle [2] \rangle$ and $\langle [4] \rangle$ are equal and therefore the orders $[2]$ and $[4]$ must be equal as well.\\
~\\
(b) Determine the order of $\alpha = 
\begin{pmatrix}
1&2&3&4\\
2&3&4&1
\end{pmatrix}
$in the group $D_4$ of symmetries of a square. What is the inverse of $\alpha$ in $D_4$ ? Directly compute the order of the inverse of $\alpha$ in $D_4$ . What do you notice?\\
~\\
First of all, we note that \[\langle \alpha \rangle = \{
\begin{pmatrix}
1&2&3&4\\
2&3&4&1
\end{pmatrix},
\begin{pmatrix}
1&2&3&4\\
3&4&1&2
\end{pmatrix},
\begin{pmatrix}
1&2&3&4\\
4&1&2&3
\end{pmatrix},
\begin{pmatrix}
1&2&3&4\\
1&2&3&4
\end{pmatrix}
\}.\] From this we see that the magnitude of $\alpha$ is $4$. The inverse of $\alpha$ is $\begin{pmatrix}
1&2&3&4\\
4&1&2&3
\end{pmatrix}
$. The cyclic group generated by $\alpha^{-1}$ is 
\[
\begin{pmatrix}
1&2&3&4\\
2&3&4&1
\end{pmatrix},
\begin{pmatrix}
1&2&3&4\\
3&4&1&2
\end{pmatrix},
\begin{pmatrix}
1&2&3&4\\
4&1&2&3
\end{pmatrix},
\begin{pmatrix}
1&2&3&4\\
1&2&3&4
\end{pmatrix}
\}.\]
There the order of $\alpha^{-1}$ is $4$. Again, this is equal to $\alpha$.\\
~\\
(c) Based on your observations in parts (a) and (b), what relationship do you
think exists between $|a|$ and $|a^{-1}|$ in a group $G$?\\
~\\
The order of $a$ is equal to the order of $a^{-1}$.\\
~\\
(d) Let $G$ be a group with identity $e$, and let $a \in G$. Show that if $a^n = e$ for
some positive integer $n$, then $(a^{-1})^n = e$.\\
~\\
\begin{proof}
Let $G$ be a group with identity $e$ and let $a \in G$ such that 
\begin{equation}\label{201}
a^n = e
\end{equation}
for some positive integer $n$. Applying the definitions of an integer power of an element in a group we see that 
\begin{equation}\label{202}
(a^{-1})^n = a^{-1\cdot n} = a^{-n} = (a^n)^{-1}
\end{equation}
Now applying the transitive property of equality to \eqref{201} and \eqref{202} we obtain
\[
(a^{-1})^n = e^{-1} = e.
\]
In conclusion, we have shown that if $G$ be a group with identity $e$ and $a \in G$ such that $a^n = e$ for some positive integer $n$, then $(a^{-1})^n = e$.
\end{proof}
(e) Let $G$ be a group with identity $e$, and let $a$ be an element of $G$ with finite
order. For this case, prove the relationship you conjectured between $|a|$
and $|a^{-1}|$ in part (c).\\
~\\
{\bf Conjecture.}  Let $G$ be a group with identity $e$ with element $a$ of finite order. The order of $a$ is equal to the order of $a^{-1}$.
\begin{proof}
Assume to the contrary that $|a|>|a^{-1}|$. We know that $\langle a \rangle$ contains $e$ and so there must exist some positive integer $n$ such that
\begin{equation}\label{20e1}
a^n = e.
\end{equation}
Let $S=\{x\in\mathbb{Z}^+:a^x=e\}$. From \eqref{20e1} we now that $S$ is not empty. Therefore from the Axiom of Choice we will choose $k$ to be the 
From the Axiom of Choice we know that there must exist some least value $k\in S$. 
\end{proof}

(f) Let $G$ be a group with identity $e$, and let $a \in G$. Prove that if $a$ has infinite
order, then $a^{-1}$ has infinite order.

\begin{proof}
Assume to the contrary that $a$ has infinite order, but $a^{-1}$ does not.
\end{proof}
~\\
(3) Let $H$ denote the set of all $2 \times 2$ matrices of the form
\[\begin{bmatrix}
x&0\\
y&0
\end{bmatrix}\]
where $x,y\in\mathbb{R}$. Is $H$ a subgroup of $\mathcal{M}_{2\times 2}(\mathbb{R})$?\\
~\\
{\bf Conjecture.} The set $H$ is a subgroup of $\mathcal{M}_{2\times 2}(\mathbb{R})$.
\begin{proof}
First of all, the identity of $\mathcal{M}_{2\times 2}(\mathbb{R})$ under addition is $\begin{bmatrix}
0&0\\
0&0
\end{bmatrix}$ and this is in $H$. Next, let $a,b,c,d\in\mathbb{R}$. Then $\begin{bmatrix}
a&0\\
b&0
\end{bmatrix}$
and $\begin{bmatrix}
c&0\\
d&0
\end{bmatrix}$ are both in $H$. When we add these two matrices together we get
\[\begin{bmatrix}
a&0\\
b&0
\end{bmatrix} + 
\begin{bmatrix}
a&0\\
b&0
\end{bmatrix} =
\begin{bmatrix}
a+c&0\\
b+d&0
\end{bmatrix}.
\]
Because the reals are closed under addition, we know that both $a+c$ and $b+d$ are in $\mathbb{R}$. Therefore, $\begin{bmatrix}
a+c&0\\
b+d&0
\end{bmatrix}$ is in $H$ and from this we can conclude that $H$ is closed under addition. Finally let $x,y\in\mathbb{R}$. The inverses of $x$ and $y$ are also in $\mathbb{R}$ and so both
$\begin{bmatrix}x&0\\y&0\end{bmatrix}$ and $\begin{bmatrix}-x&0\\-y&0\end{bmatrix}$ are in $H$. We also see that
\[\begin{bmatrix}x&0\\y&0\end{bmatrix}+\begin{bmatrix}-x&0\\-y&0\end{bmatrix}=\begin{bmatrix}0&0\\0&0\end{bmatrix}.\] From this, we can conclude that every element in $H$ has an inverse. In conclusion, we have shown that the the subset $H$ of $\mathcal{M}_{2\times 2}(\mathbb{R})$ is closed under addition, the identity of $\mathcal{M}_{2\times 2}(\mathbb{R})$ under addition is in $H$, and every element of $H$ has an inverse. From this we have shown that $H$ is a subgroup of $\mathcal{M}_{2\times 2}(\mathbb{R})$ under addition.
\end{proof}
~\\
(4) Let $G$ be a group and $H$ a subgroup of $G$. Which of the following conjectures do you think are true, and which do you think are false? Provide brief arguments or examples to justify your answers.\\
(a) If $G$ is finite, then $H$ is finite.\\
~\\
This statement is true due to the fact that $H$ is a subset of $G$ and must therefore cannot have a magnitude greater than its superset $G$.\\
~\\
(b) If $H$ is finite, then $G$ is finite.\\
~\\
This is not necessarily true. For example, $\{0\}$ is a subgroup of $\mathbb{Z}$ under addition. In this case $H$ is finite, but $G$ is infinite.\\
~\\
(c) If $G$ is Abelian, then $H$ is Abelian.\\
~\\
This is a property of the operator of $G$ and therefore it will also be true for $H$ whose elements are all in $G$.\\
~\\
(d) If $H$ is Abelian, then $G$ is Abelian.\\
~\\
This is not always true. For example, we have the group $G$ containing the symmetries of a square with the subgroup $H$ containing only its identity. In this case $H$ is Abelian, but $G$ is not.


\end{document}