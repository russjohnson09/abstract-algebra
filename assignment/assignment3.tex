\documentclass[11pt,a4paper]{article}

\usepackage{amsmath}
\usepackage{amsfonts}
\usepackage{amssymb}
\usepackage{amsthm}

\usepackage{graphicx}

\usepackage{verbatim}

\usepackage{hyperref}


%no paragraph indent
\setlength{\parindent}{0pt}


\begin{document}

\begin{flushright}
Russ Johnson\\
Problem Set $\#3$\\
\today\\
\end{flushright}

{\bf (15)} Let $\mathcal{F}(\mathbb{R})$ denote the set of all functions from $\mathbb{R}$ to $\mathbb{R}$. Define addition and multiplication on $\mathcal{F}(\mathbb{R})$ as follows:
\begin{itemize}
\item For all $f,g \in \mathcal{F}(\mathbb{R}), (f + g) :\mathbb{R} \rightarrow \mathbb{R}$ is the function defined by
\[(f+g)(x) = f(x)+g(x)\]
for all $x\in\mathbb{R}$.
\item $f,g \in \mathcal{F}(\mathbb{R}), (fg) :\mathbb{R} \rightarrow \mathbb{R}$ is the function defined by
\[(fg)(x) = f(x)g(x)\]
for all $x\in\mathbb{R}$.
\end{itemize}
~\\
(a) Prove that $\mathcal{F}(\mathbb{R})$ is an Abelian group under addition.\\
~\\
For this proof we will first show that $\mathcal{F}(\mathbb{R})$ is closed under addition, that addition is associative in $\mathcal{F}(\mathbb{R})$, that $\mathcal{F}(\mathbb{R})$ contains an identity element, and that each element in $\mathcal{F}(\mathbb{R})$ has an inverse. Finally, we will show that addition is commutative in $\mathcal{F}(\mathbb{R})$.\\
~\\
First of all, from the definition of addition in $\mathcal{F}(\mathbb{R})$ we see that this operation will always give us another function from the reals to the reals which is also an element in $\mathcal{F}(\mathbb{R})$. Therefore, $\mathcal{F}(\mathbb{R})$ is closed under addition. Next, we will prove that  addition is associative in $\mathcal{F}(\mathbb{R})$.
\begin{proof}
Let $f,g,h\in\mathcal{F}(\mathbb{R})$ and let $x\in\mathbb{R}$. From the definition of addition in $\mathcal{F}(\mathbb{R})$ we know that
\begin{equation}\label{151}
((f+g)+h)(x) = (f(x)+g(x))+h(x)
\end{equation}
and
\begin{equation}\label{152}
(f+(g+h))(x) = f(x)+(g(x)+h(x))
\end{equation}
We also know that
\begin{equation}\label{153}
f(x) \in \mathbb{R},
\end{equation}
\begin{equation}\label{154}
g(x) \in \mathbb{R},
\end{equation}
and
\begin{equation}\label{155}
h(x) \in \mathbb{R}
\end{equation}
from the fact that the codomain of $f,g,$ and $h$ is $\mathbb{R}$.
We already know that addition is associative in $\mathbb{R}$. From this fact and the equations \eqref{153}, \eqref{154}, and \eqref{155} we also know that
\begin{equation}\label{156}
(f(x)+g(x))+h(x) = f(x)+(g(x)+h(x)).
\end{equation}
Applying the transitive property of equality to \eqref{156} and \eqref{151}, we obtain
\begin{equation}\label{157}
((f+g)+h)(x) = f(x)+(g(x)+h(x)).
\end{equation}
Applying this same property again to \eqref{157} and \eqref{152}, we obtain
\begin{equation}\label{158}
((f+g)+h)(x) = (f+(g+h))(x).
\end{equation}
Finally, from \eqref{158} we see that addition is associative in $\mathcal{F}(\mathbb{R})$.
\end{proof}
Next we will prove that $\mathcal{F}(\mathbb{R})$ contains an identity element.
\begin{proof}
Let $e\in\mathcal{F}(\mathbb{R})$ such that 
\begin{equation}
e(x) = 0
\end{equation}
for all $x\in\mathbb{R}$. Let $f\in\mathcal{F}(\mathbb{R})$ and let $a\in\mathbb{R}$. We see that
\[(f+e)(a) = f(a)+e(a)=f(a)+0=f(a)\]
and
\[(e+f)(a) = e(a)+f(a)=0+f(a)=f(a).\]
In conclusion, we have shown that $(f+e)(a)=(e+f)(a)=f(a)$ for some $a\in\mathbb{R}$ and therefore the function $e$ defined $e(x)=0$ for all $x\in\mathbb{R}$ is the identity element in $\mathcal{F}(\mathbb{R})$.
\end{proof}
Next will show that each function in $\mathcal{F}(\mathbb{R})$ has an inverse under addition.
\begin{proof}
Let $f\in\mathcal{F}(\mathbb{R})$ and let $x\in\mathbb{R}$. First we note that $f(x)\in\mathbb{R}$. Because the reals are a group under addition we know that there exits some $g\in\mathcal{F}(\mathbb{R})$ such that $f(x)+g(x)=0$ and $g(x)+f(x)=0$. Therefore, the arbitrary element $f\in\mathcal{F}(\mathbb{R})$ has an inverse.\\
~\\
In conclusion we have shown that every element in $\mathcal{F}(\mathbb{R})$ has an inverse.
\end{proof}
Finally, we will show that every element in $\mathcal{F}(\mathbb{R})$ commutes under addition.
\begin{proof}
Let $f,g\in\mathcal{F}(\mathbb{R})$ and let $x\in\mathbb{R}$. We know that
\begin{equation}\label{15e1}
(f+g)(x)= f(x)+g(x)
\end{equation}
and
\begin{equation}\label{15e2}
(g+f)(x)=g(x)+f(x)
\end{equation}
from the definition of a function in $\mathcal{F}(\mathbb{R})$. We also know that
\begin{equation}\label{15e3}
f(x)\in\mathbb{R}
\end{equation}
and
\begin{equation}\label{15e4}
g(x)\in\mathbb{R}
\end{equation}
from the fact that a function in $\mathcal{F}(\mathbb{R})$ has the codomain $\mathbb{R}$. Because addition is commutative in $\mathbb{R}$, from \eqref{15e3} and \eqref{15e4} we can conclude that
\begin{equation}\label{15e5}
f(x)+g(x)=g(x)+f(x).
\end{equation}
Applying the transitive property of equality to \eqref{15e1} and \eqref{15e5} we obtain
\begin{equation}\label{15e6}
(f+g)(x)=g(x)+f(x).
\end{equation}
Applying this same property to \eqref{15e6} and \eqref{15e2} we obtain
\[(g+f)(x)=(f+g)(x).\]
In conclusion, we have shown that $(g+f)(x)=(f+g)(x)$ for all $x\in\mathbb{R}$ and therefore addition is commutative in $\mathcal{F}(\mathbb{R})$.
\end{proof}
~\\
In conclusion, we have shown that $\mathcal{F}(\mathbb{R})$ is closed under addition, addition is associative in $\mathcal{F}(\mathbb{R})$, $\mathcal{F}(\mathbb{R})$ contains an identity element, each element in $\mathcal{F}(\mathbb{R})$ has an inverse, and addition is commutative in $\mathcal{F}(\mathbb{R})$. From this we know that $\mathcal{F}(\mathbb{R})$ is an Abelian group under addition.\\
~\\
(b) Does $\mathcal{F}(\mathbb{R})$ have an identity element for multiplication?\\
~\\
Yes, let $e\in\mathcal{F}(\mathbb{R})$ such that
\[e(x)=1\]
for all $x\in\mathbb{R}$. Let $f\in\mathcal{F}(\mathbb{R})$ and let $a\in\mathbb{R}$. We see that
\[(fe)(a)=f(a)e(a)=f(a)\cdot 1=f(a)\]
and
\[(ef)(a)=e(a)f(a)=1\cdot f(a)=f(a).\]
Therefore, $e$ is the identity in $\mathcal{F}(\mathbb{R})$ under multiplication.\\
~\\
(c) Find an element in $\mathcal{F}(\mathbb{R})$ that does not have a multiplicative inverse in $\mathcal{F}(\mathbb{R})$. Explain how this shows $\mathcal{F}(\mathbb{R})$ is not a group under multiplication.\\
~\\
Let $f\in\mathcal{F}(\mathbb{R})$ such that
\[f(x)=0\]
for all $x\in\mathbb{R}$. Let $g\in\mathcal{F}(\mathbb{R})$ and let $a\in\mathbb{R}$. We see that
\[(fg)(a)=0\cdot g(a)=0\neq 1.\]
And from this we can conclude that the function $f$ defined as $f(x)=0$ for all $x\in\mathbb{R}$ has no inverse. Because a group must have an inverse for every element and $f$ is an element in $\mathcal{F}(\mathbb{R})$, $\mathcal{F}(\mathbb{R})$ is not a group under multiplication.\\
~\\
(d) Find necessary and sufficient conditions for an element in $\mathcal{F}(\mathbb{R})$ to be a unit in $\mathcal{F}(\mathbb{R})$. State your result in a lemma of the form ``The function $f\in\mathcal{F}(\mathbb{R})$ is a unit in $\mathcal{F}(\mathbb{R})$ if and only if ...''. Your lemma must say something more than just a rehash of the definition of a unit; rather, it must actually characterize the functions that are invertible under multiplication in $\mathcal{F}(\mathbb{R})$.\\
~\\
{\bf Conjecture.} An element in $\mathcal{F}(\mathbb{R})$ is a unit if and only if $f(x)\neq 0$ for all $x\in\mathbb{R}$.\\
~\\
First we will show that if $f$ is a function in $\mathcal{F}(\mathbb{R})$ such that $f(x)\neq 0$ for all $x\in\mathbb{R}$, then $f$ is a unit.
\begin{proof}
Let $f$ be a function in $\mathcal{F}(\mathbb{R})$ such that $f(x)\neq 0$ for all $x\in\mathbb{R}$ and let $a\in\mathbb{R}$. Because the set $\mathbb{R}-\{0\}$ is a group under multiplication and $f(x)$ is in this group we know that there exists some $g$ in $\mathcal{F}(\mathbb{R})$ such that $f(x)g(x)=1$ and $g(x)f(x)=1$. From this we see that $f$ has an inverse in $\mathcal{F}(\mathbb{R})$.
\end{proof}
Next we will show that if $f$ is a unit, then it must be true that $f(x)\neq 0$ for all $x\in\mathbb{R}$.
\begin{proof}
We assume to the contrary that $f$ is a unit in $\mathcal{F}(\mathbb{R})$ such that there exists some $a\in\mathbb{R}$ such that $f(a)=0$. Because $f$ is a unit, there must exist some $g$ in $\mathcal{F}(\mathbb{R})$ such that $g(x)f(x)=1$ for all $x\in\mathbb{R}$. This implies that there exists a real number $g(a)$ such that $g(a)\cdot 0 =1$. This is a contradiction and so it must be necessary for $f(x)\neq 0$ for all $x\in\mathbb{R}$ in order for $f$ to be a unit.
\end{proof}
~\\
{\bf Activity 20.12.} In this activity, we will explore a simple relationship between the order of a group element and the order of its inverse.\\
~\\
(a) Determine the order of $[2]$ in $\mathbb{Z}_6$. What is the inverse of $[2]$ in $\mathbb{Z}_6$? Directly
compute the order of the inverse of $[2]$ in $\mathbb{Z}_6$. What do you notice?\\
~\\
First of all, we note that $\langle [2] \rangle = \{[0],[2],[4]\}$. The magnitude of this set is $3$ and therefore the order of $[2]$ in $\mathbb{Z}_6$ is $3$. The inverse of $[2]$ is $[4]$ ($[2]+[4]=[0]$). The order of $[4]$ in $\mathbb{Z}_6$ is equal to the magnitude of $\langle [4] \rangle = \{[0],[2],[4]\}$, and so the order of $[4]$ is $3$. The sets $\langle [2] \rangle$ and $\langle [4] \rangle$ are equal and therefore the orders $[2]$ and $[4]$ must be equal as well.\\
~\\
(b) Determine the order of $\alpha = 
\begin{pmatrix}
1&2&3&4\\
2&3&4&1
\end{pmatrix}
$in the group $D_4$ of symmetries of a square. What is the inverse of $\alpha$ in $D_4$ ? Directly compute the order of the inverse of $\alpha$ in $D_4$ . What do you notice?\\
~\\
First of all, we note that \[\langle \alpha \rangle = \left\{
\begin{pmatrix}
1&2&3&4\\
2&3&4&1
\end{pmatrix},
\begin{pmatrix}
1&2&3&4\\
3&4&1&2
\end{pmatrix},
\begin{pmatrix}
1&2&3&4\\
4&1&2&3
\end{pmatrix},
\begin{pmatrix}
1&2&3&4\\
1&2&3&4
\end{pmatrix}
\right\}.\] From this we see that the magnitude of $\alpha$ is $4$. The inverse of $\alpha$ is $\begin{pmatrix}
1&2&3&4\\
4&1&2&3
\end{pmatrix}
$. The cyclic group generated by $\alpha^{-1}$ is 
\[
\left\{\begin{pmatrix}
1&2&3&4\\
2&3&4&1
\end{pmatrix},
\begin{pmatrix}
1&2&3&4\\
3&4&1&2
\end{pmatrix},
\begin{pmatrix}
1&2&3&4\\
4&1&2&3
\end{pmatrix},
\begin{pmatrix}
1&2&3&4\\
1&2&3&4
\end{pmatrix}
\right\}.\]
Therefore the order of $\alpha^{-1}$ is $4$. Again, this is equal to $\alpha$.\\
~\\
(c) Based on your observations in parts (a) and (b), what relationship do you
think exists between $|a|$ and $|a^{-1}|$ in a group $G$?\\
~\\
The order of $a$ is equal to the order of $a^{-1}$.\\
~\\
(d) Let $G$ be a group with identity $e$, and let $a \in G$. Show that if $a^n = e$ for
some positive integer $n$, then $(a^{-1})^n = e$.\\
~\\
\begin{proof}
Let $G$ be a group with identity $e$ and let $a \in G$ such that 
\begin{equation}\label{201}
a^n = e
\end{equation}
for some positive integer $n$. Applying the definitions of an integer power of an element in a group we see that 
\begin{equation}\label{202}
(a^{-1})^n = a^{-1\cdot n} = a^{-n} = (a^n)^{-1}
\end{equation}
Now applying the transitive property of equality to \eqref{201} and \eqref{202} we obtain
\[
(a^{-1})^n = e^{-1} = e.
\]
In conclusion, we have shown that if $G$ be a group with identity $e$ and $a \in G$ such that $a^n = e$ for some positive integer $n$, then $(a^{-1})^n = e$.
\end{proof}
(e) Let $G$ be a group with identity $e$, and let $a$ be an element of $G$ with finite
order. For this case, prove the relationship you conjectured between $|a|$
and $|a^{-1}|$ in part (c).\\
~\\
{\bf Conjecture.}  Let $G$ be a group with identity $e$ with element $a$ of finite order. The order of $a$ is equal to the order of $a^{-1}$.
\begin{proof}
We know that $\langle a \rangle$ contains $e$ and so there must exist some positive integer $n$ such that
\begin{equation}\label{20e1}
a^n = e.
\end{equation}
Let $S=\{x\in\mathbb{Z}^+:a^x=e\}$. From \eqref{20e1} we know that $S$ is not empty. Therefore, from the Axiom of Choice we are able to choose the a smallest element $k\in S$. The set $\{a,a^2,\ldots,a^k\}$ is equal to the set $\langle a \rangle$ (it contains $a$ and all of the elements up to $e$) and so the order of $a$ is $k$. From part (d) we know that $(a^{-1})^k=e$ and so $|a^{-1}|\leq |a|$.\\
~\\
We also know that $\langle a^{-1} \rangle$ contains $e$ and so there must exist some positive integer $m$ such that
\begin{equation}\label{20e2}
(a^{-1})^m = e.
\end{equation}
Let $T=\{x\in\mathbb{Z}^+:((a)^{-1})^x=e\}$. From \eqref{20e2} we know that $T$ is not empty. Therefore, from the Axiom of Choice we are able to choose the a smallest element $p\in T$. The set $\{a^{-1},(a^{-1})^2,\ldots,(a^{-1})^p\}$ is equal to the set $\langle a^{-1} \rangle$ (it contains $a^{-1}$ and all of the elements up to $e$) and so the order of $a^{-1}$ is $p$. From part (d) we know that $a^p=e$ and so $|a|\leq |a^{-1}|$.\\
~\\
In conclusion we have shown that the order of $a$ is less than or equal to the order of $a^{-1}$ and the order of $a^{-1}$ is less than or equal to the order of $a$. Therefore, the order of $a$ and the order of $a^{-1}$ is equal.
\end{proof}

(f) Let $G$ be a group with identity $e$, and let $a \in G$. Prove that if $a$ has infinite
order, then $a^{-1}$ has infinite order.

\begin{proof}
Assume to the contrary that $a$ has infinite order, but $a^{-1}$ does not. Because the order of $a^{-1}$ is finite, there must exist some $n\in\mathbb{Z}^+$ such that
\begin{equation}\label{f1}
|a^{-1}| < n
\end{equation}
The inverse of $a^n$ is $a^{-1})^n$, but because the order of $(a^{-1}$ is less than $n$ there must exist some $k<n$ such that $(a^{-1})^n=(a^{-1})^k$. But this would mean that $(a^{-1})^k$ has two unique inverses, $a^k$ and $a^n$. This is a contradiction and so $a^{-1}$ has infinite order.
\end{proof}
~\\
(3) Let $H$ denote the set of all $2 \times 2$ matrices of the form
\[\begin{bmatrix}
x&0\\
y&0
\end{bmatrix}\]
where $x,y\in\mathbb{R}$. Is $H$ a subgroup of $\mathcal{M}_{2\times 2}(\mathbb{R})$?\\
~\\
{\bf Conjecture.} The set $H$ is a subgroup of $\mathcal{M}_{2\times 2}(\mathbb{R})$.
\begin{proof}
First of all, the identity of $\mathcal{M}_{2\times 2}(\mathbb{R})$ under addition is $\begin{bmatrix}
0&0\\
0&0
\end{bmatrix}$ and this is in $H$ ($x=0,y=0$). Next, let $a,b,c,d\in\mathbb{R}$. Then $\begin{bmatrix}
a&0\\
b&0
\end{bmatrix}$
and $\begin{bmatrix}
c&0\\
d&0
\end{bmatrix}$ are both in $H$. When we add these two matrices together we get
\[\begin{bmatrix}
a&0\\
b&0
\end{bmatrix} + 
\begin{bmatrix}
a&0\\
b&0
\end{bmatrix} =
\begin{bmatrix}
a+c&0\\
b+d&0
\end{bmatrix}.
\]
Because the reals are closed under addition, we know that both $a+c$ and $b+d$ are in $\mathbb{R}$. Therefore, $\begin{bmatrix}
a+c&0\\
b+d&0
\end{bmatrix}$ is in $H$ and from this we can conclude that $H$ is closed under addition. Finally let $x,y\in\mathbb{R}$. The inverses of $x$ and $y$ are also in $\mathbb{R}$ and so both
$\begin{bmatrix}x&0\\y&0\end{bmatrix}$ and $\begin{bmatrix}-x&0\\-y&0\end{bmatrix}$ are in $H$. We also see that
\[\begin{bmatrix}x&0\\y&0\end{bmatrix}+\begin{bmatrix}-x&0\\-y&0\end{bmatrix}=\begin{bmatrix}0&0\\0&0\end{bmatrix}.\] From this, we can conclude that every element in $H$ has an inverse. In conclusion, we have shown that the the subset $H$ of $\mathcal{M}_{2\times 2}(\mathbb{R})$ is closed under addition, the identity of $\mathcal{M}_{2\times 2}(\mathbb{R})$ under addition is in $H$, and every element of $H$ has an inverse. From this we have shown that $H$ is a subgroup of $\mathcal{M}_{2\times 2}(\mathbb{R})$ under addition.
\end{proof}
~\\
(4) Let $G$ be a group and $H$ a subgroup of $G$. Which of the following conjectures do you think are true, and which do you think are false? Provide brief arguments or examples to justify your answers.\\
(a) If $G$ is finite, then $H$ is finite.\\
~\\
This statement is true due to the fact that $H$ is a subset of $G$ and therefore cannot have a magnitude greater than its superset $G$.\\
~\\
(b) If $H$ is finite, then $G$ is finite.\\
~\\
This is not necessarily true. For example, $\{0\}$ is a subgroup of $\mathbb{Z}$ under addition. In this case $H$ is finite, but $G$ is infinite.\\
~\\
(c) If $G$ is Abelian, then $H$ is Abelian.\\
~\\
This is a property of the operator of $G$ and therefore it will also be true for $H$ whose elements are all in $G$.\\
~\\
(d) If $H$ is Abelian, then $G$ is Abelian.\\
~\\
This is not always true. For example, we have the group $G$ containing the symmetries of a square with the subgroup $H$ containing only its identity. In this case $H$ is Abelian, but $G$ is not.\\
~\\
(8) {\bf Intersections of subgroups.} Let $G$ be a group with subgroups $H$ and $K$.\\
(a) Is $H \cap K$ a subgroup of $G$? Prove your answer.\\
~\\
{\bf Conjecture.} Let $G$ be a group with subgroups $H$ and $K$. Then $H \cap K$ is a subgroup of $G$.
\begin{proof}
Let $G$ be a group with identity $e$ and let $H$ and $K$ be subgroups of $G$. We know from the fact that $H$ and $K$ are subgroups of $G$ that $e\in H$ and $e\in K$, and so $e\in H\cap K$. Next, let $a,b\in H\cap K$. From the definition of intersection of sets we know that $a,b\in H$ and $a,b\in K$. Because $H$ and $K$ are groups, $ab\in H$ and $ab\in K$. Therefore $ab\in H\cap K$, coming from the definition of intersection. From this we have shown that $H\cap K$ are closed under the operation of $G$. Finally, let $x\in H\cap K$. From the definition of intersection of sets we know that $x\in H$ and $x\in K$. Because $H$ and $K$ are groups, $x^{-1}\in H$ and $x^{-1}\in K$. Therefore $x^{-1}\in H\cap K$, coming from the definition of intersection of sets.\\
~\\
In conclusion, we have shown that the identity of $G$ is in $H\cap K$, the set $H\cap K$ is closed under the operation of $G$, and each element in $H\cap K$ has an inverse in $H\cap K$. Therefore, $H\cap K$ is a subgroup of $G$.
\end{proof}
~\\
(b) Can we generalize? That is, if ${H_\alpha}$ is a collection of subgroups of $G$ indexed by $\alpha$ in an indexing set $I$, is it the case that $\cap_{\alpha\in I} H_\alpha$ is a subgroup of $G$? Prove your answer.\\
~\\
{\bf Conjecture.} Let ${H_\alpha}$ be a collection of subgroups of $G$ indexed by $\alpha$ in an indexing set $I$. Then $\cap_{\alpha\in I} H_\alpha$ is a subgroup of $G$.\\
\begin{proof}
Let $H$ be a subgroup of $G$ in the collection $H_\alpha$. We know that $e\in H$ from the fact that $H$ is a subgroup of $G$. Because $H$ is an arbitrary element in $H_\alpha$ we have shown that $e$ is in every element of $H_\alpha$. Therefore, from the definition of intersection of sets $e\in\cap_{\alpha\in I} H_\alpha$. Next let $a,b\in\cap_{\alpha\in I} H_\alpha$. From the definition of intersection we know that $a,b\in H$. Because $H$ is a group, $ab\in H$. Again, $H$ is an arbitrary element in $H_\alpha$ and so $ab\in\cap_{\alpha\in I} H_\alpha$. From this, we see that $\cap_{\alpha\in I} H_\alpha$ is closed under the operation of the group $G$. Finally, let $x\in \cap_{\alpha\in I} H_\alpha$. From the definition of intersection we know that $x\in H$. Because $H$ is a group, $x^{-1}\in H$. And because $H$ is an arbitrary element in $H_\alpha$, $x^{-1}\in\cap_{\alpha\in I} H_\alpha$.\\
~\\
In conclusion, we have shown that $\cap_{\alpha\in I} H_\alpha$ contains the identity $e\in G$, it is closed under the operation of $G$, and each element in $\cap_{\alpha\in I} H_\alpha$ has an inverse. Therefore, $\cap_{\alpha\in I} H_\alpha$ is a subgroup of $G$.
\end{proof}
~\\
(12) Determine whether $H$ is a subgroup of $G$.\\
(a) $G=\mathbb{Z}_{20}$ under addition, $H=\{[0],[3],[6],[9],[12],[15],[18]\}$.\\
~\\
The inverse of $[3]$ is $[17]$ which is not in $H$. Therefore $H$ is not a subgroup of $G$.\\
~\\
(b) $G=U_7$ under multiplication, $H=\{[1],[2],[4]\}$.\\
~\\
First of all, the identity $[1]$ is in $H$. Now we will construct an operation table to see if $H$ is closed under multiplication and each element of $H$ has an inverse.
\[
\begin{array}{c|c|c|c}
\cdot & [1] & [2] & [4]\\\hline
[1] & [1] & [2] & [4]\\\hline
[2] & [2] & [4] & [1]\\\hline
[4] & [4] & [1] & [2]
\end{array}
\]
From this operation table we see that $H$ is closed under multiplication and each element of $H$ has an inverse. And so, we can conclude that $H$ is a subgroup of $G$.\\
~\\
(c) $G=U_{16}$ and $H=\{[1],[7],[9],[15]\}$.\\
~\\
First of all, the identity $[1]$ is in $H$. Now we will construct an operation table to see if $H$ is closed under multiplication and each element of $H$ has an inverse.
\[
\begin{array}{c|c|c|c|c}
\cdot & [1] & [7] & [9] & [15] \\\hline
[1] & [1] & [7] & [9] & [15]\\ \hline
[7] & [7] & [1] & [15] & [19] \\ \hline
[9] & [9] & [15] & [1] & [7] \\ \hline
[15] & [15] & [9] & [7] & [1]
\end{array} 
\]
From this operation table we see that $H$ is closed under multiplication and each element of $H$ has an inverse. And so, we can conclude that $H$ is a subgroup of $G$.\\

\end{document}