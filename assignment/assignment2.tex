\documentclass[11pt,a4paper]{article}

\usepackage{amsmath}
\usepackage{amsfonts}
\usepackage{amssymb}
\usepackage{amsthm}

\usepackage{graphicx}

\usepackage{verbatim}

\usepackage{hyperref}


%no paragraph indent
\setlength{\parindent}{0pt}


\begin{document}

\begin{flushright}
Russ Johnson\\
Problem Set $\#2$\\
\today\\
\end{flushright}

{\bf Activity 18.11}\\
(a) In a group $G$ with identity $e$, if $ab = e$ for some $a, b \in G$ must it follow
that $b = a^{-1}$?

{\bf Conjecture. } In a group $G$ with identity $e$, if $ab = e$ for some $a, b \in G$, then $ba = e$ and consequently $b = a^{-1}$.

\begin{proof}
Let $G$ be a group with identity $e$ and let $a,b \in G$ such that
\begin{equation}\label{11}
ab = e.
\end{equation}
Multiplying $a$ on the right side of \eqref{11} we obtain
\begin{equation}\label{12}
(ab)a = ea.
\end{equation}
Applying the associative property of groups from \eqref{12} we know that
\begin{equation}\label{13}
a(ba) = ea.
\end{equation}
Because $e$ is the identity of $G$,
\begin{equation}\label{14}
ea = ae.
\end{equation}
Applying the transitive property of equality to \eqref{13} and \eqref{14} we obtain
\begin{equation}\label{15}
a(ba) = ae.
\end{equation}
Applying the group cancellation law to \eqref{15} we obtain
\begin{equation}
ba = e.
\end{equation}

In conclusion we have shown that in a group $G$ with identity $e$, if $ab = e$ for some $a, b \in G$, then $ba = e$. From the fact that $ab = e$ and $ba = e$ we can conclude that $b$ is the inverse of $a$.
\end{proof}

(a) In a group $G$ with identity $e$, if $ba = e$ for some $a, b \in G$ must it follow
that $b = a^{-1}$?

{\bf Conjecture. } In a group $G$ with identity $e$, if $ba = e$ for some $a, b \in G$, then $ab = e$ and consequently $b = a^{-1}$.

\begin{proof}
Let $G$ be a group with identity $e$ and let $a,b \in G$ such that
\begin{equation}\label{b1}
ba = e.
\end{equation}
Multiplying $a$ on the left side of \eqref{b1} we obtain
\begin{equation}\label{b2}
a(ba) = ae.
\end{equation}
Applying the associative property of groups from \eqref{b2} we know that
\begin{equation}\label{b3}
(ab)a = ae.
\end{equation}
Because $e$ is the identity of $G$,
\begin{equation}\label{b4}
ae = ea.
\end{equation}
Applying the transitive property of equality to \eqref{b3} and \eqref{b4} we obtain
\begin{equation}\label{b5}
(ab)a = ea.
\end{equation}
Applying the group cancellation law to \eqref{b5} we obtain
\begin{equation}
ab = e.
\end{equation}

In conclusion we have shown that in a group $G$ with identity $e$, if $ba = e$ for some $a, b \in G$, then $ab = e$. From the fact that $ab = e$ and $ba = e$ we can conclude that $b$ is the inverse of $a$.
\end{proof}

(c) Let $f$ and $g$ be functions from a set $S$ to $S$. Let $I$ be the identity function
on $S$ -- that is $I(x) = x$ for all $x$ in $S$. Show by example that it is possible
to have $fg = I$, but $f \neq g^{-1}$. Does this violate part (a)? Explain.

Let $f$ and $g$ be functions from the set $\mathbb{C}$ to $\mathbb{C}$ defined as $f(x)=x^2$ and $g(x)=\sqrt{x}$. Now $f\circ g (x) = x$, however $g\circ f (x) = |x|$ and so $f$ is not the inverse of $g$. This fact does not violate what we found in part (a), because this statement applies only to groups and $g$ is not a group.

(4) Determine if the set $G$ is a group under the indicated operation. If $G$ is
a group, verify that each group property is satisfied. If $G$ is not a group,
provide examples that show which of the group properties are not satisfied.\\
~\\
(a) Let $G$ be the set of odd integers under addition.\\
~\\
The set $G$ is not a group. The identity element for this set must be zero, but zero is not an odd integer. Therefore, $G$ does not have an identity element and is not a group.\\
~\\
(b) Let $G = {[2], [4], [6], [8]} \subset \mathbb{Z}_{10}$, with the operation of multiplication
of congruence classes.\\
~\\
First we will construct an operation table for this group.
\begin{center}
$
\begin{array}{c|c|c|c|c}
\cdot & [2] & [4] & [6] & [8] \\\hline
[2] & [4] & [8] & [2] & [6] \\\hline
[4] & [8] & [6] & [4] & [2] \\\hline
[6] & [2] & [4] & [6] & [8] \\\hline
[8] & [6] & [2] & [8] & [4] \\
\end{array}
$
\end{center}
~\\
From this operation table we see that $a\cdot [6] = [6]\cdot a = a$ for all $a\in G$ and so $[6]$ is $G$'s identity element. We can also see from the operation table that for all $a \in G$ there exits a $b\in G$ such that $a\cdot b = b\cdot a = [6]$ and so every element has an inverse. There are no elements in the operation table that are not in $G$ and so $G$ is closed under multiplication of congruence classes. We already know that multiplication of congruence classes is associative and so it will be associative in $G$.\\
~\\
In conclusion, we have shown that the set $G$ has an identity, is closed, is associative, and each element has an inverse under multiplication of congruence classes. Therefore, $G$ is a group under multiplication of congruence classes.\\
~\\
(c) Let $G = {[0], [2], [4], [6], [8]} \subset \mathbb{Z}_{10}$, with the operation of addition of
congruence classes.\\
~\\
First we will construct an operation table for this group.
\begin{center}
$
\begin{array}{c|c|c|c|c|c}
+ & [0] & [2] & [4] & [6] & [8] \\\hline
[0] & [0] & [2] & [4] & [6] & [8] \\\hline
[2] & [2] & [4] & [6] & [8] & [0] \\\hline
[4] & [4] & [6] & [8] & [0] & [2] \\\hline
[6] & [6] & [8] & [0] & [2] & [4] \\\hline
[8] & [8] & [0] & [2] & [4] & [6] \\
\end{array}
$
\end{center}
~\\
From this operation table we see that $a + [0] = [0] + a = a$ for all $a\in G$ and so $[0]$ is $G$'s identity element. We can also see from the operation table that for all $a \in G$ there exits a $b\in G$ such that $a + b = b + a = [0]$ and so every element has an inverse. There are no elements in the operation table that are not in $G$ and so $G$ is closed under addition of congruence classes. We already know that addition of congruence classes is associative and so it will be associative in $G$.\\
~\\
In conclusion, we have shown that the set $G$ has an identity, is closed, is associative, and each element has an inverse under addition of congruence classes. Therefore, $G$ is a group under addition of congruence classes.\\
~\\
(d) Let  $G = {q \in Q : q \neq 1}$, with the operation $*$ defined by $a * b = a + b - ab$.\\
~\\
First we not that $a * 0 = 0 * a = a$ for all $a\in G$ and so $0$ is the identity element in $G$. We know that integers are associative under addition and this operation can be defined using only addition. Therefore $G$ is associative under the operator $*$. Next we will show that $G$ is closed.
\begin{proof}
Let $a,b \in G$ such that
\begin{equation}\label{21}
a+b - ab = 1
\end{equation}
We are working with integers under addition and subtraction which we now is commutative and associative. We subtract $b$ from \eqref{21} to obtain
\begin{equation}\label{22}
a-ab = 1-b.
\end{equation}
Factoring out $a$ on the left side of \eqref{22} we obtain
\begin{equation}\label{23}
a(1-b) = 1-b
\end{equation}
From \eqref{22} we can apply the group cancellation law to arrive at the contradiction
\[a = 1.\]
From this contradiction, we know that $G$ is closed under the operation $*$. 
\end{proof}
~\\
Finally we will show that every element has an inverse. Let $a,b \in G$ such that
\[b = \dfrac{a}{a-1}.\] The element $b$ is only undefined when $a = 1$ and there is no $a$ such that $b=1$. We can also see that \[a+b-ab = b+a-ba = a + \dfrac{a}{a-1} - \dfrac{a\cdot a}{a-1} = 0.\]
From this we see that each element $a\in G$ has an inverse and this inverse is $\dfrac{a}{a-1}$.\\
~\\
In conclusion, we have shown that the set $G$ has an identity, is closed, is associative, and each element has an inverse under the operator $*$. Therefore, $G$ is a group under the operator $*$.\\
~\\
(e) Let $G = {[x] \in \mathbb{Z}_9 : x = 1, 2, 4, 5, 7, \text{ or } 8}$, with the operation $[x] *
[y] = [x][y]$.\\
~\\
First we will construct an operation table.
\begin{center}
$
\begin{array}{c|c|c|c|c|c|c}
* & [1] & [2] & [4] & [5] & [7] & [8]\\\hline
[1] & [1] & [2] & [4] & [5] & [7] & [8]\\\hline
[2] & [2] & [4] & [8] & [1] & [5] & [7]\\\hline
[4] & [4] & [8] & [7] & [2] & [1] & [5]\\\hline
[5] & [5] & [1] & [2] & [7] & [8] & [4]\\\hline
[7] & [7] & [5] & [1] & [8] & [4] & [2]\\\hline
[8] & [8] & [7] & [5] & [4] & [2] & [1]\\
\end{array}
$
\end{center}
~\\
From this operation table we see that $a*[1] = [1]*a = a$ for all $a\in G$ and so $[1]$ is $G$'s identity element. We can also see from the operation table that for all $a \in G$ there exits a $b\in G$ such that $a*b = b*a = [1]$ and so every element has an inverse. There are no elements in the operation table that are not in $G$ and so $G$ is closed under the binary operator $*$. We already know that multiplication of congruence classes is associative and so it will be associative in $G$.\\
~\\
In conclusion, we have shown that the set $G$ has an identity, is closed, is associative, and each element has an inverse under the operator $*$. Therefore, $G$ is a group under the operator $*$.\\
~\\
(7) Let $k$ be an integer, and let $Z(k)$ be the set of integers on which an operation
$\oplus_k$ is defined as follows:
$a \oplus_k b = a + b − k$,
where $a + b$ denotes the standard sum of $a$ and $b$ in $\mathbb{Z}$. Note that the set $Z(0)$
is the group of integers under the standard addition. For which values of $k$ is
$Z(k)$ a group under the operation $\oplus_k$?\\
~\\
{\bf Conjecture. } The set $Z(k)$ is a group for all $k\in \mathbb{Z}$.\\
~\\
First of all we note that $a+k-k = k + a -k = a$, and so $k$ is the identity element. This is consistent with the fact that $0$ is the identity element of integers under addition, $Z(0)$. Next we note that the operation $\oplus$ can be defined using only addition and the set of all integers is associative under addition. Therefore $Z(k)$ is associative for all $k\in \mathbb{Z}$. The set of integers are closed under addition and subtraction and so $Z(k)$ is closed under $\oplus_k$, because it consists of only adding and subtracting integers. Finally, every element $a\in Z(k)$ has an inverse. We see that $a + -a -k = -a + a - k = k$ and so the inverse of the arbitrary element $a \in Z(k)$ is $-a$.\\
~\\
In conclusion, we have shown that $Z(k)$ has an identity element, is associative, is closed, and each element has an inverse for all $k\in \mathbb{Z}$. Therefore $Z(k)$ is a group for all $k\in \mathbb{Z}$.\\
~\\
(8) Prove that a group $G$ is Abelian if and only if $(ab)^2 = a^2 b^2$ for all $a, b \in G$.

\begin{proof}
First we will show that if a group $G$ is Abelian, then $(ab)^2 = a^2 b^2$ for all $a, b \in G$.\\
~\\
Let $G$ be an abelian group and let $a,b\in G$. First of all,
\begin{equation}\label{31}
(ab)^2 = (ab)(ab).
\end{equation}
First, we apply the associative property of the group $G$ to \label{31} to obtain
\begin{equation}\label{32}
(ab)^2 = a(ba)b.
\end{equation}
Next, we use the commutative property of $G$ on \eqref{32} to obtain
\begin{equation}\label{33}
(ab)^2 = a(ab)b.
\end{equation}
Finally, applying the associative property to \eqref{33} we obtain
\[(ab)^2 = a^2b^2.\]
~\\
Next we will show that if $(ab)^2 = a^2b^2$ for all $a,b\in G$, then $G$ is Abelian.
~\\
Let $G$ be a group such that
\begin{equation}\label{34}
(ab)^2 = a^2b^2
\end{equation}
for all $a,b \in G$. From \eqref{34} we also know that
\begin{equation}\label{35}
(ab)(ab) = (aa)(bb).
\end{equation}
Applying the associative property to \eqref{35} we obtain
\begin{equation}\label{36}
a((ba)b) = a((ab)b).
\end{equation}
Applying the group cancellation law to \eqref{36} we know that
\begin{equation}\label{37}
(ba)b = (ab)b.
\end{equation}
Applying the group cancellation law to \eqref{37} we obtain
\[ba = ab.\]
From this we can conclude that $G$ is Abelian.\\
~\\
In conclusion, we have shown that if a group $G$ is Abelian, then $(ab)^2 = a^2 b^2$ for all $a, b \in G$. We have also shown that if $(ab)^2 = a^2b^2$ for all $a,b\in G$, then $G$ is Abelian. Therefore we have shown that a group $G$ is Abelian if and only if $(ab)^2 = a^2 b^2$ for all $a, b \in G$.
\end{proof}



\end{document}

Guidelines:
1. The statement of each problem must be written in its entirety, followed by the solution.

2. If you obtain an idea from someone else credit them.

3. Write in complete sentences.

4. Avoid run-ons.

5. Do not begin a sentence with a symbol.

6. As a writer, it is your job to make things clear to the reader. Do not use the words "clearly" or obviously.

7. Proofread. Use a spell-checker and be sure to re-read everything at least two times for accuracy and clarity before submitting.


Understand your audience.

Steve Schlicker