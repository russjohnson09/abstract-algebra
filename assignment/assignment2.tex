\documentclass[11pt,a4paper]{article}

\usepackage{amsmath}
\usepackage{amsfonts}
\usepackage{amssymb}
\usepackage{amsthm}

\usepackage{graphicx}

\usepackage{verbatim}

\usepackage{hyperref}


%no paragraph indent
\setlength{\parindent}{0pt}


\begin{document}

\begin{flushright}
Russ Johnson\\
Problem Set $\#2$\\
\today\\
\end{flushright}

{\bf Activity 18.11}\\
(a) In a group $G$ with identity $e$, if $ab = e$ for some $a, b \in G$ must it follow
that $b = a^{-1}$?\\
~\\
{\bf Conjecture.} In a group $G$ with identity $e$, if $ab = e$ for some $a, b \in G$, then $ba = e$ and consequently $b = a^{-1}$.

\begin{proof}
Let $G$ be a group with identity $e$ and let $a,b \in G$ such that
\begin{equation}\label{11}
ab = e.
\end{equation}
Multiplying $a$ on the right side of \eqref{11} we obtain
\begin{equation}\label{12}
(ab)a = ea.
\end{equation}
Applying the associative property of groups from \eqref{12} we know that
\begin{equation}\label{13}
a(ba) = ea.
\end{equation}
Because $e$ is the identity of $G$ we can commute $e$ in \eqref{13} to obtain
\begin{equation}\label{15}
a(ba) = ae.
\end{equation}
Applying the group cancellation law to \eqref{15} we obtain
\begin{equation}
ba = e.
\end{equation}

In conclusion we have shown that in a group $G$ with identity $e$, if $ab = e$ for some $a, b \in G$, then $ba = e$. From the fact that $ab = e$ and $ba = e$ we can conclude that $b$ is the inverse of $a$.
\end{proof}

(b) In a group $G$ with identity $e$, if $ba = e$ for some $a, b \in G$ must it follow
that $b = a^{-1}$?\\
~\\
{\bf Conjecture.} In a group $G$ with identity $e$, if $ba = e$ for some $a, b \in G$, then $ab = e$ and consequently $b = a^{-1}$.

\begin{proof}
Let $G$ be a group with identity $e$ and let $a,b \in G$ such that
\begin{equation}\label{b1}
ba = e.
\end{equation}
Multiplying $a$ on the left side of \eqref{b1} we obtain
\begin{equation}\label{b2}
a(ba) = ae.
\end{equation}
Applying the associative property of groups from \eqref{b2} we know that
\begin{equation}\label{b3}
(ab)a = ae.
\end{equation}
Because $e$ is the identity of $G$ we can commute $e$ in \eqref{b3} to obtain
\begin{equation}\label{b5}
(ab)a = ea.
\end{equation}
Applying the group cancellation law to \eqref{b5} we obtain
\begin{equation}
ab = e.
\end{equation}

In conclusion we have shown that in a group $G$ with identity $e$, if $ba = e$ for some $a, b \in G$, then $ab = e$. From the fact that $ab = e$ and $ba = e$ we can conclude that $b$ is the inverse of $a$.
\end{proof}

(c) Let $f$ and $g$ be functions from a set $S$ to $S$. Let $I$ be the identity function
on $S$ -- that is $I(x) = x$ for all $x$ in $S$. Show by example that it is possible
to have $fg = I$, but $f \neq g^{-1}$. Does this violate part (a)? Explain.\\
~\\
Let $f$ and $g$ be functions from the set $\mathbb{C}$ to $\mathbb{C}$ defined as $f(x)=x^2$ and $g(x)=\sqrt{x}$. Now $f\circ g (x) = x$, however $g\circ f (x) = |x|$ and so $f$ is not the inverse of $g$. This fact does not violate what we found in part (a), because this statement applies only to groups and $f$ is not in a group. For the function $f$, $f(2) = f(-2) = 4$ and so it cannot have an inverse, because a function would only be able to map $4$ to $2$ or to $-2$ and not both. From this we see that $f$ is not in a group and so what we found in parts (a) and (b) does not apply here.\\
~\\
(4) Determine if the set $G$ is a group under the indicated operation. If $G$ is
a group, verify that each group property is satisfied. If $G$ is not a group,
provide examples that show which of the group properties are not satisfied.\\
~\\
(a) Let $G$ be the set of odd integers under addition.\\
~\\
The set $G$ is not a group. The identity element for this set must be zero because it is a subset of the integers, but zero is not an odd integer. Therefore, $G$ does not have an identity element and is not a group.\\
~\\
(b) Let $G = \{[2], [4], [6], [8]\} \subset \mathbb{Z}_{10}$, with the operation of multiplication
of congruence classes.\\
~\\
First we will construct an operation table for this group.
\begin{center}
$
\begin{array}{c|c|c|c|c}
\cdot & [2] & [4] & [6] & [8] \\\hline
[2] & [4] & [8] & [2] & [6] \\\hline
[4] & [8] & [6] & [4] & [2] \\\hline
[6] & [2] & [4] & [6] & [8] \\\hline
[8] & [6] & [2] & [8] & [4] \\
\end{array}
$
\end{center}
~\\
From this operation table we see that $a\cdot [6] = [6]\cdot a = a$ for all $a\in G$ and so $[6]$ is $G$'s identity element. We can also see from the operation table that for all $a \in G$ there exits some $b\in G$ such that $a\cdot b = b\cdot a = [6]$ and so every element in $G$ has an inverse. There are no elements in the operation table that are not in $G$ and so $G$ is closed under multiplication of congruence classes. We already know that multiplication of congruence classes is associative in the set $\mathbb{Z}_{10}$ and so it will be associative in its subset $G$.\\
~\\
In conclusion, we have shown that the set $G$ has an identity, is closed, is associative, and each element has an inverse under multiplication of congruence classes. Therefore, $G$ is a group under multiplication of congruence classes.\\
~\\
(c) Let $G = \{[0], [2], [4], [6], [8]\} \subset \mathbb{Z}_{10}$, with the operation of addition of
congruence classes.\\
~\\
First we will construct an operation table for this group.
\begin{center}
$
\begin{array}{c|c|c|c|c|c}
+ & [0] & [2] & [4] & [6] & [8] \\\hline
[0] & [0] & [2] & [4] & [6] & [8] \\\hline
[2] & [2] & [4] & [6] & [8] & [0] \\\hline
[4] & [4] & [6] & [8] & [0] & [2] \\\hline
[6] & [6] & [8] & [0] & [2] & [4] \\\hline
[8] & [8] & [0] & [2] & [4] & [6] \\
\end{array}
$
\end{center}
~\\
From this operation table we see that $a + [0] = [0] + a = a$ for all $a\in G$ and so $[0]$ is $G$'s identity element. We can also see from the operation table that for all $a \in G$ there exits some $b\in G$ such that $a + b = b + a = [0]$ and so every element in $G$ has an inverse. There are no elements in the operation table that are not in $G$ and so $G$ is closed under addition of congruence classes. We already know that addition of congruence classes is associative in the set $\mathbb{Z}_{10}$ and so it will be associative in its subset $G$.\\
~\\
In conclusion, we have shown that the set $G$ has an identity, is closed, is associative, and each element has an inverse under addition of congruence classes. Therefore, $G$ is a group under addition of congruence classes.\\
~\\
(d) Let  $G = {q \in Q : q \neq 1}$, with the operation $*$ defined by $a * b = a + b - ab$.\\
~\\
Let $a\in G$. First we note that $a * 0 = a + 0 - a\cdot 0 = a$ and $0 * a = 0 + a - 0\cdot a = a$ for all $a\in G$ and so $0$ is the identity element in $G$. Let $b\in G$. We know that rational numbers are associative under addition, subtraction, and multiplication and so the operator $*$ will be associative in the set $\mathbb{Q}$. Therefore, $G\subset \mathbb{Q}$ is associative under the operator $*$. Next we will show that $G$ is closed under $*$.
\begin{proof}
We will prove that $G$ is closed under $*$ by contradiction. Assume there exist some $a$ and $b$ in $G$ such that
\begin{equation}\label{21}
a+b - ab = 1
\end{equation}
We are working with rational numbers under addition, subtraction, and multiplication and so we can apply some simple algebra. We add the inverse of $b$ in \eqref{21} to obtain
\begin{equation}\label{22}
a-ab = 1 + -b.
\end{equation}
Applying the distributive property of multiplication over subtraction to \eqref{22} we obtain
\begin{equation}\label{23}
a(1-b) = 1-b
\end{equation}
From \eqref{22} we can apply the group cancellation law to arrive at the contradiction
\[a = 1.\]
From this contradiction, we know that $G$ is closed under the operator $*$. 
\end{proof}
~\\
Finally we will show that every element has an inverse. Let $a\in G$. The inverse of $a$ is $b\in G$ such that
\[b = \dfrac{a}{a-1}.\] The element $b$ is only undefined when $a = 1$ and there is no $a$ such that $b=1$. We can also see that \[a+b-ab = b+a-ba = a + \dfrac{a}{a-1} - \dfrac{a\cdot a}{a-1} = 0.\]
From this we see that every element in $G$ has an inverse.\\
~\\
In conclusion, we have shown that the set $G$ has an identity, is closed, is associative, and each element has an inverse under the operator $*$. Therefore, $G$ is a group under the operator $*$.\\
~\\
(e) Let $G = \{[x] \in \mathbb{Z}_9 : x = 1, 2, 4, 5, 7, \text{ or } 8\}$, with the operation $[x] *
[y] = [x][y]$.\\
~\\
First we will construct an operation table.
\begin{center}
$
\begin{array}{c|c|c|c|c|c|c}
* & [1] & [2] & [4] & [5] & [7] & [8]\\\hline
[1] & [1] & [2] & [4] & [5] & [7] & [8]\\\hline
[2] & [2] & [4] & [8] & [1] & [5] & [7]\\\hline
[4] & [4] & [8] & [7] & [2] & [1] & [5]\\\hline
[5] & [5] & [1] & [2] & [7] & [8] & [4]\\\hline
[7] & [7] & [5] & [1] & [8] & [4] & [2]\\\hline
[8] & [8] & [7] & [5] & [4] & [2] & [1]\\
\end{array}
$
\end{center}
~\\
From this operation table we see that $a*[1] = [1]*a = a$ for all $a\in G$ and so $[1]$ is $G$'s identity element. We can also see from the operation table that for all $a \in G$ there exits some $b\in G$ such that $a*b = b*a = [1]$ and so every element has an inverse. There are no elements in the operation table that are not in $G$ and so $G$ is closed under the binary operator $*$. We already know that multiplication of congruence classes is associative in $\mathbb{Z}_9$ and so it will be associative in its subset $G$.\\
~\\
In conclusion, we have shown that the set $G$ has an identity, is closed, is associative, and each element has an inverse under the operator $*$. Therefore, $G$ is a group under the operator $*$.\\
~\\
(7) Let $k$ be an integer, and let $Z(k)$ be the set of integers on which an operation
$\oplus_k$ is defined as follows:
$a \oplus_k b = a + b - k$, where $a + b$ denotes the standard sum of $a$ and $b$ in $\mathbb{Z}$. Note that the set $Z(0)$ is the group of integers under the standard addition. For which values of $k$ is $Z(k)$ a group under the operation $\oplus_k$?\\
~\\
{\bf Conjecture.} The set $Z(k)$ is a group for all $k\in \mathbb{Z}$.\\
~\\
First of all we note that $a+k-k = k + a -k = a$, and so $k$ is the identity element. This is consistent with the fact that $0$ is the identity element of integers under addition, $Z(0)$. Next we note that the operation $\oplus$ can be defined using only addition as $a \oplus_k b = a + b + -k$. We already know that the set of integers is associative under addition and so the set of integers is associate under $\oplus_k$ for all $k\in \mathbb{Z}$. The set of integers are closed under addition and subtraction and so $\mathbb{Z}$ is closed under $\oplus_k$, because it consists of only addition and subtraction. Finally, every element $a\in Z(k)$ has an inverse. We see that $a + -a -k = -a + a - k = k$ and so the inverse of the arbitrary element $a \in Z(k)$ is $-a$.\\
~\\
In conclusion, we have shown that $Z(k)$ has an identity element, is associative, is closed, and each element has an inverse for all $k\in \mathbb{Z}$. Therefore $Z(k)$ is a group for all $k\in \mathbb{Z}$.\\
~\\
(8) Prove that a group $G$ is Abelian if and only if $(ab)^2 = a^2 b^2$ for all $a, b \in G$.\\
~\\
\begin{proof}
First we will show that if a group $G$ is Abelian, then $(ab)^2 = a^2 b^2$ for all $a, b \in G$.\\
~\\
Let $G$ be an abelian group and let $a,b\in G$. First of all,
\begin{equation}\label{31}
(ab)^2 = (ab)(ab).
\end{equation}
First, we apply the associative property of the group $G$ to \label{31} to obtain
\begin{equation}\label{32}
(ab)^2 = a(ba)b.
\end{equation}
Next, we use the commutative property of $G$ on \eqref{32} to obtain
\begin{equation}\label{33}
(ab)^2 = a(ab)b.
\end{equation}
Finally, applying the associative property to \eqref{33} we obtain
\[(ab)^2 = a^2b^2.\]
~\\
Next we will show that if $(ab)^2 = a^2b^2$ for all $a,b\in G$, then $G$ is Abelian.
~\\
Let $G$ be a group such that
\begin{equation}\label{34}
(ab)^2 = a^2b^2
\end{equation}
for all $a,b \in G$. From \eqref{34} we also know that
\begin{equation}\label{35}
(ab)(ab) = (aa)(bb).
\end{equation}
Applying the associative property to \eqref{35} we obtain
\begin{equation}\label{36}
a((ba)b) = a((ab)b).
\end{equation}
Applying the group cancellation law to \eqref{36} we know that
\begin{equation}\label{37}
(ba)b = (ab)b.
\end{equation}
Applying the group cancellation law to \eqref{37} we obtain
\[ba = ab.\]
From this we can conclude that $G$ is Abelian.\\
~\\
In conclusion, we have shown that if a group $G$ is Abelian, then $(ab)^2 = a^2 b^2$ for all $a, b \in G$. We have also shown that if $(ab)^2 = a^2b^2$ for all $a,b\in G$, then $G$ is Abelian. Therefore we have shown that a group $G$ is Abelian if and only if $(ab)^2 = a^2 b^2$ for all $a, b \in G$.
\end{proof}
~\\
(1) Let $G$ be a group.\\
(a) Let $a, b, c \in G$. What element is $(abc)^{-1}$ ?\\
~\\
{\bf Conjecture.} The inverse of $abc$ is $c^{-1}b^{-1}a^{-1}$.

\begin{proof}
Let $G$ be a group and let $a, b, c \in G$. First of all, we know that
\begin{equation}\label{41}
(abc)(c^{-1}b^{-1}a^{-1}) = (abc)(c^{-1}b^{-1}a^{-1})
\end{equation}
from the reflexive property of equality.
Applying the associative property to \eqref{41} we obtain
\begin{equation}\label{42}
(abc)(c^{-1}b^{-1}a^{-1}) = (ab(cc^{-1})(b^{-1}a^{-1}).
\end{equation}
Multiplying $c$ and the inverse of $c$ in \eqref{42} we obtain the identity element of $G$, $e$.
\begin{equation}\label{43}
(abc)(c^{-1}b^{-1}a^{-1}) = (abe)(b^{-1}a^{-1}).
\end{equation}
The identity element can be multiplied out of \eqref{43} to obtain
\begin{equation}
(abc)(c^{-1}b^{-1}a^{-1}) = (ab)(b^{-1}a^{-1}).
\end{equation}
Repeating this process we see that
\[(abc)(c^{-1}b^{-1}a^{-1}) = a(bb^{-1})a^{-1} = aea^{-1} = aa^{-1} = e.\]
We can also apply the same rules to show that 
\[(c^{-1}b^{-1}a^{-1})(abc) = (c^{-1}b^{-1})(bc) = c^{-1}c = e.\]
In conclusion, we have shown that 
\[(c^{-1}b^{-1}a^{-1})(abc) = (abc)(c^{-1}b^{-1}a^{-1} = e\]
and therefore the inverse of $abc$ is $c^{-1}b^{-1}a^{-1}$.
\end{proof}

(b) Let $m$ be a positive integer, and let $a_1, a_2 ,\ldots, a_m$ be elements in $G$.
What element is $(a_1 a_2 \cdots a_m )^{-1}$ ?\\
~\\
{\bf Conjecture.} The inverse of $a_1 a_2 \ldots a_m$ is $a_m^{-1}a_{m-1}^{-1}\ldots a_1^{-1}$ for all $m\in\mathbb{Z^+}$.

\begin{proof}
We will prove our conjecture using induction. First we note that for $m=1$ we have $a_1^{-1} = a_1^{-1}$ and for $m=2$ we have $(a_1a_2)^{-1} = a_2^{-1}a_1^{-1}$. The equation for $m=1$ is self-evident. For $m=2$ we see that
\[(a_1a_2)(a_2^{-1}a_1^{-1}) = a_1(a_2a_2^{-1})a_1^{-1} = a_1ea_1^{-1} = a_1a_1^{-1} = e\]
and
\[(a_2^{-1}a_1^{-1})(a_1a_2) = a_2^{-1}(a_1^{-1}a_1)a_2 = a_2^{-1}ea_2 = a_2^{-1}a_2 = e.\]
And so, for $m=2$ the conjecture is true.
~\\
For the next part of the proof we will show that
\[(a_1 a_2 \ldots a_m)^{-1} = a_m^{-1}a_{m-1}^{-1}\ldots a_1^{-1} \]
implies that
\[(a_1 a_2 \ldots a_m a_{m+1})(a_{m+1}^{-1} a_m^{-1}\ldots a_1^{-1}) = e\]
and
\[(a_{m+1}^{-1} a_m^{-1}\ldots a_1^{-1})(a_1 a_2 \ldots a_m a_{m+1}) = e\]
and therefore the inverse of  $a_1 a_2 \ldots a_m a_{m+1}$ is $a_{m+1}^-1 a_m^{-1}a_{m-1}^{-1}\ldots a_1^{-1}$. First of all,
\begin{equation}\label{51}
(a_1 a_2 \ldots a_m a_{m+1})(a_{m+1}^{-1} a_m^{-1}\ldots a_1^{-1}) = (a_1 a_2 \ldots a_m a_{m+1})(a_{m+1}^{-1} a_m^{-1}\ldots a_1^{-1}).
\end{equation}
Applying the associative property to \eqref{51} we obtain
\begin{equation}\label{52}
(a_1 a_2 \ldots a_m a_{m+1})(a_{m+1}^{-1} a_m^{-1}\ldots a_1^{-1}) = (a_1 a_2 \ldots a_m) (a_{m+1}a_{m+1}^{-1})( a_m^{-1}\ldots a_1^{-1}).
\end{equation}
Multiplying out $a_{m+1}a_{m+1}^{-1}$ in \eqref{52} we obtain
\begin{equation}\label{53}
(a_1 a_2 \ldots a_m a_{m+1})(a_{m+1}^{-1} a_m^{-1}\ldots a_1^{-1}) = (a_1 a_2 \ldots a_m)( a_m^{-1}\ldots a_1^{-1}).
\end{equation}
We know that
\begin{equation}\label{54}
(a_1 a_2 \ldots a_m)( a_m^{-1}\ldots a_1^{-1}) = e
\end{equation}
from the hypothesis of the inductive proof. Applying the transitive property of equality to \eqref{53} and \eqref{54} we obtain
\begin{equation}
(a_1 a_2 \ldots a_m a_{m+1})(a_{m+1}^{-1} a_m^{-1}\ldots a_1^{-1}) = e.
\end{equation}
In a similar manner we can show that
\[(a_{m+1}^{-1} a_m^{-1}\ldots a_1^{-1})(a_1 a_2 \ldots a_m a_{m+1}) = a_{m+1}^{-1}ea_{m+1} =  a_{m+1}^{-1}a_{m+1} = e. \]

From this we see can conclude that the inverse of $a_1 a_2 \ldots a_m$ is $a_m^{-1}a_{m-1}^{-1}\ldots a_1^{-1}$ for all $m\in\mathbb{Z^+}$.


\end{proof}

(2) Prove that if $G$ is a group with identity $e$ in which $a^2 = e$ for every $a \in G$, then $G$ is an Abelian group. Is the converse true?\\
~\\
\begin{proof}
Let $G$ be a group with identity $e$ such that $a^2 = e$ for all $a\in G$. Let $a,b\in G$. We know that
\begin{equation}\label{61}
(ab)(ab) = e.
\end{equation}
from the fact that $G$ must be closed under its operation and so $ab\in G$. 
Multiplying $a$ on the left side of \eqref{61} we obtain
\begin{equation}\label{62}
a(ab)(ab) = ae.
\end{equation}
Because $e$ is the identity element in $G$ we know that $ae = a$. Applying this and the transitive property of equality to \eqref{62} we obtain
\begin{equation}\label{63}
a(ab)(ab) = a.
\end{equation}
Applying the associative property to \eqref{63} we obtain
\begin{equation}\label{64}
a^2b(ab) = a.
\end{equation}
From the knowledge that $a^2= e$, \eqref{64} becomes
\begin{equation}\label{65}
b(ab) = a.
\end{equation}
Multiplying $b$ on the right side of \eqref{65} we obtain
\begin{equation}\label{66}
b(ab)b = ab.
\end{equation}
Applying the associative property to \eqref{66} we obtain
\begin{equation}\label{67}
(ba)b^2 = ab.
\end{equation}
Again, coming from the fact that any element multiplied by itself is the identity element, \eqref{67} becomes
\[ba = ab.\]
~\\
In conclusion we have shown that $G$ is Abelian.
\end{proof}

The converse of the statement in problem 2 is not true. For example we have the Abelian group $(\mathbb{Z},+)$. In this group $2+2 \neq 0$ and so being an Abelian group does not guarantee that for every element $a$ in the group with identity $e$, $a^2 = e$.

\end{document}