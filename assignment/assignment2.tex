\documentclass[11pt,a4paper]{article}

\usepackage{amsmath}
\usepackage{amsfonts}
\usepackage{amssymb}
\usepackage{amsthm}

\usepackage{graphicx}

\usepackage{verbatim}

\usepackage{hyperref}


%no paragraph indent
\setlength{\parindent}{0pt}


\begin{document}

\begin{flushright}
Russ Johnson\\
Problem Set $\#2$\\
\today\\
\end{flushright}

{\bf Activity 18.11}\\
(a) In a group $G$ with identity $e$, if $ab = e$ for some $a, b \in G$ must it follow
that $b = a^{-1}$?

{\bf Conjecture. } In a group $G$ with identity $e$, if $ab = e$ for some $a, b \in G$, then $ba = e$ and consequently $b = a^{-1}$.

\begin{proof}
Let $G$ be a group with identity $e$ and let $a,b \in G$ such that
\begin{equation}\label{11}
ab = e.
\end{equation}
Multiplying $a$ on the right side of \eqref{11} we obtain
\begin{equation}\label{12}
(ab)a = ea.
\end{equation}
Applying the associative property of groups from \eqref{12} we know that
\begin{equation}\label{13}
a(ba) = ea.
\end{equation}
Because $e$ is the identity of $G$,
\begin{equation}\label{14}
ea = ae.
\end{equation}
Applying the transitive property of equality to \eqref{13} and \eqref{14} we obtain
\begin{equation}\label{15}
a(ba) = ae.
\end{equation}
Applying the group cancellation law to \eqref{15} we obtain
\begin{equation}
ba = e.
\end{equation}

In conclusion we have shown that in a group $G$ with identity $e$, if $ab = e$ for some $a, b \in G$, then $ba = e$. From the fact that $ab = e$ and $ba = e$ we can conclude that $b$ is the inverse of $a$.
\end{proof}

(a) In a group $G$ with identity $e$, if $ba = e$ for some $a, b \in G$ must it follow
that $b = a^{-1}$?

{\bf Conjecture. } In a group $G$ with identity $e$, if $ba = e$ for some $a, b \in G$, then $ab = e$ and consequently $b = a^{-1}$.

\begin{proof}
Let $G$ be a group with identity $e$ and let $a,b \in G$ such that
\begin{equation}\label{b1}
ba = e.
\end{equation}
Multiplying $a$ on the left side of \eqref{b1} we obtain
\begin{equation}\label{b2}
a(ba) = ae.
\end{equation}
Applying the associative property of groups from \eqref{b2} we know that
\begin{equation}\label{b3}
(ab)a = ae.
\end{equation}
Because $e$ is the identity of $G$,
\begin{equation}\label{b4}
ae = ea.
\end{equation}
Applying the transitive property of equality to \eqref{b3} and \eqref{b4} we obtain
\begin{equation}\label{b5}
(ab)a = ea.
\end{equation}
Applying the group cancellation law to \eqref{b5} we obtain
\begin{equation}
ab = e.
\end{equation}

In conclusion we have shown that in a group $G$ with identity $e$, if $ba = e$ for some $a, b \in G$, then $ab = e$. From the fact that $ab = e$ and $ba = e$ we can conclude that $b$ is the inverse of $a$.
\end{proof}

(c) Let $f$ and $g$ be functions from a set $S$ to $S$. Let $I$ be the identity function
on $S$ -- that is $I(x) = x$ for all $x$ in $S$. Show by example that it is possible
to have $fg = I$, but $f \neq g^{-1}$. Does this violate part (a)? Explain.

Let $f$ and $g$ be functions from the set $\mathbb{C}$ to $\mathbb{C}$ defined as $f(x)=x^2$ and $g(x)=\sqrt{x}$. Now $f\circ g (x) = x$, however $g\circ f (x) = |x|$ and so $f$ is not the inverse of $g$. This fact does not violate what we found in part (a), because this statement applies only to groups and $g$ is not a group.

\end{document}

Guidelines:
1. The statement of each problem must be written in its entirety, followed by the solution.

2. If you obtain an idea from someone else credit them.

3. Write in complete sentences.

4. Avoid run-ons.

5. Do not begin a sentence with a symbol.

6. As a writer, it is your job to make things clear to the reader. Do not use the words "clearly" or obviously.

7. Proofread. Use a spell-checker and be sure to re-read everything at least two times for accuracy and clarity before submitting.


Understand your audience.

Steve Schlicker