\documentclass[11pt,a4paper]{article}

\usepackage{amsmath}
\usepackage{amsfonts}
\usepackage{amssymb}

\usepackage{hyperref}

%no paragraph indent
\setlength{\parindent}{0pt}

\begin{document}
\begin{flushright}
Russ Johnson\\
Problem Set $\#1$\\
\today\\
\end{flushright}

{\bf Activity 17.4} Find, via a library or internet search, an object (building, tiling,
painting, sculpture, mosaic, fractal, rug, etc.) that has significant symmetry. Then
complete the following.

(a) Identify the object and the source through which you found it. Choose something other than a simple polygon; that is, find an object that is interesting to you and that possesses at least 6 symmetries, including both rotational and reflective symmetry.

The $K_5$ complete graph contains all of the same symmetries as a regular pentagon. It has a total of ten symmetries.\\
~\\
\begin{flushright}
source: \url{enwikipedia.org/wiki/File:4-simplex_graph.svg}
\end{flushright}

b) Describe all of the symmetries possessed by your object. Choose 6 symmetries (including at least one non-trivial rotation and one non-trivial reflection), and make a copy of the picture of your object for each symmetry.
Find a convenient way to label your object so that you can use permutation notation to represent each symmetry. Then illustrate each symmetry on one of the copies of your picture.

The $K_5$ graph when drawn as it is from my source has reflective symmetries 

c) Choose 3 of the symmetries, and find all of the compositions of these three symmetries. Is each composition a symmetry of your object? Explain.

(3) 

(a) Find all of the symmetries of the letter B, and create the operation table
for the set of symmetries of B.

(b) Find all of the symmetries of the letter T, and create the operation table
for the set of symmetries of T.

(c) Find all of the symmetries of the letter Z, and create the operation table
for the set of symmetries of Z.

(d) Compare the operation tables for B, T, and Z. Describe all of the similarities and differences you observe.

(4) Is composition of symmetries a commutative operation? Prove your answer.


(5) Let A, B, and C be the objects shown in Figure 17.5.

(a) For each object, find all of the symmetries. Describe the symmetries in words and using the permutation notation introduced in this investigation.

(b) Create the operation table for the set of symmetries of each object.

(c) Describe the similarities and differences in the operation tables you made in part (b). Your description should include not only obvious attributes like the number of elements, but also how the elements interact within a given set of symmetries.


{\bf Activity 18.12.} Recall that a symmetry of an object $O$ is a bijective, distance preserving function f such that $f (O) = O$. In this activity, we will verify that the set S of symmetries of an object O forms a group under the operation of composition, called the group of symmetries of $O$.

(a) Let f and g be bijective, distance-preserving functions with f (O) = O
and $g(O) = O$. To show that $S$ is closed, we need to verify that $f \circ g$ is a
bijective, distance preserving function with $(f \circ g)(O) = O$.

(i) Prove that $|(f \circ g)(x) − (f \circ g)(y)| = |x − y|$ for all x, y in the
domain of $f \circ g$.

(ii) Use part (a) to deduce that $f \circ g$ is an injection.

(iii) Show that $(f \circ g)(O) = O$. Deduce that $f \circ g$ is a surjection.

(iv) Explain how parts (a)–(c) establish that S is closed under composition.

(b) Prove that composition of functions is an associative operation.

(c) What is the identity element of S? Verify your answer.

(d) If f ∈ S, what is the inverse of f in S? Verify your answer. (Please note
that there is quite a bit to do to complete this problem.)

(e) Explain why S is a group. Is S an Abelian group? Explain.

\end{document}

Guidelines:
1. The statement of each problem must be written in its entirety, followed by the solution.

2. If you obtain an idea from someone else credit them.

3. Write in complete sentences.

4. Avoid run-ons.

5. Do not begin a sentence with a symbol.

6. As a writer, it is your job to make things clear to the reader. Do not use the words "clearly" or obviously.

7. Proofread. Use a spell-checker and be sure to re-read everything at least two times for accuracy and clarity before submitting.


Understand your audience.

Steve Schlicker