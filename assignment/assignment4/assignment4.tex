\documentclass[11pt,a4paper]{article}

\usepackage{amsmath}
\usepackage{amsfonts}
\usepackage{amssymb}
\usepackage{amsthm}

\usepackage{graphicx}

\usepackage{verbatim}

\usepackage{hyperref}


%no paragraph indent
\setlength{\parindent}{0pt}


\begin{document}

\begin{flushright}
Russ Johnson\\
Problem Set $\#4$\\
\today\\
\end{flushright}

(4)\\
{ (a)} Show that $U_{22}$ is cyclic.\\
~\\
The congruence class $[7]$ is a generator of $U_{22}$ and therefore $U_{22}$ is cyclic.
\[U_{22} = \langle [7] \rangle = \{[7], [5],[13],[3],[21],[15],[17],[9],[19],[1]\}.\]
~\\
{ (b)} Find all the generators of $U_{22}$. Explain how you know that each element is a generator.\\
~\\
We know that $U_{22}$ is a group and therefore is closed under its operation. Therefore, by finding an element in $U_{22}$ with the same order as $U_{22}$, we are guaranteed that this is a generator of $U_{22}$. The order of $U_{22}$ is $10$ and from part $(a)$ we already have the generator $[7]$ with order $10$. From Theorem 21.3 part (ii) we know that the order of $[7]^k$ for some positive integer $k$ is equal to $\dfrac{10}{gcd(k,10)}$. Therefore, when $k$ and $10$ are coprime, we know that the order of $[7]^k$ is $10$ and is a generator of $U_{22}$. And so, the generators of $U_{22}$ are $[7]^3 = [13]$, $[7]^7 = [17]$, and $[7]^9 = [19]$.\\
~\\
{(5)} Let $A = \begin{bmatrix}1&0\\0&-1\end{bmatrix} $ and $B = \begin{bmatrix}1&1\\0&-1\end{bmatrix}$.\\
{(a)} Find $|A|$ and $|B|$.\\
~\\
We note that $\langle A \rangle =\left\{ \begin{bmatrix}1&0\\0&-1\end{bmatrix}, \begin{bmatrix}1&0\\0&1\end{bmatrix} \right\}$ and  $\langle B \rangle = \left\{ \begin{bmatrix}1&1\\0&-1\end{bmatrix}, \begin{bmatrix}1&0\\0&1\end{bmatrix} \right\}$. And so, $|A| = |B| = 2$.
~\\
(b) Determine $|AB|$. Does your answer surprise you? Explain.\\
~\\
First of all 
\[AB = \begin{bmatrix}1&1\\0&1\end{bmatrix}, \]
\[AB^2 = \begin{bmatrix}1&2\\0&1\end{bmatrix},\]
and
\[AB^3 = \begin{bmatrix}1&3\\0&1\end{bmatrix}.\]
From this we see that \[\langle AB \rangle = \left\{ \begin{bmatrix}1&a\\0&1\end{bmatrix} | a \in \mathbb{N}   \right\}.\]
And so, $AB$ has infinite order. This answer is somewhat surprising, but by multiplying $A$ and $B$ we obtain a matrix that is not in $\langle A \rangle$ or $\langle B \rangle$ and so we cannot expect it to be finite.
~\\
(9) Prove Theorem 21.5.
\begin{proof}
Let $G = \langle a \rangle$ be an infinite cyclic group with identity $e$, and let $b \neq e$ be an element in $G$. Let $m,n\in \mathbb{Z}^+$ such that 
\begin{equation}\label{91}
b^m = b^n.
\end{equation}
The element $a$ is the generator of $G$ and therefore there must exist some positive integer $k$ such that 
\begin{equation}\label{92}
a^k = b.
\end{equation}
Raising both sides of equation \eqref{92} to the power of $m$ we obtain
\begin{equation}\label{93}
a^{km} = b^m.
\end{equation}
Raising both sides of equation \eqref{92} to the power of $n$ we obtain
\begin{equation}\label{94}
a^{kn} = b^n.
\end{equation}
From equations \eqref{91}, \eqref{93}, and \eqref{94} we know that
\begin{equation}\label{95}
a^{kn} = a^{km}.
\end{equation}
The element $a$ has infinite order and so from \eqref{95} we know that
\begin{equation}\label{96}
kn = km.
\end{equation}
Applying the Group Cancellation Rule to \eqref{96} we obtain
\[n = m.\]
In conclusion, we have shown that if $b^m = b^n$ for some positive integer powers $m$ and $n$, then $m = n$. The contrapositive of this is that all integer powers of $b$ are distinct. And so by proving this fact we have shown that $\langle b \langle$ is an infinite cyclic group. The fact that $\langle b \langle$, is a cyclic group comes from the Theorem 21.1 and the unique powers proof means that $b$ has infinite order.
\end{proof}
~\\
(18) Let $G$ be a group and let $a,b\in G$ with $|a| = b$ and $|b| = m$.\\
(a) Is it necessarily true that $|ab| = mn$?\\
(b) If $ab = ba$, is it necessarily true that $|ab| = mn$?\\
~\\
This counter-example works for parts (a) and (b). The congruence classes $[7]$ and $[7]$ are both in $U_{22}$. We have already shown that $[7]$ has an order of $10$ and $U_{22}$ also has an order of $10$. It is for an element in $U_{22}$ to have an order of $100$, which is greater than the order of the group.\\
~\\
(c) Prove that if $ab = ba$ and $gcd(m,n) = 1$, then the order of $ab$ is $mn$.

\begin{proof}
First of all we have must show that $ab$ has finite order. We know that $ab = ba$ and so
\[(ab)^{nm} = a^{nm}b^{nm} = (a^n)^m = (b^m)^n = e^m e^n = e.\]
From this we see that $ab$ does have finite order. Let $q$ be the order of $ab$. By Theorem 21.2 (ii) we know that $q | nm$. From Theorem 21.2 (i) we know that
\begin{equation}\label{181}
(ab)^q = e.
\end{equation}
Raising both sides of equation \eqref{181} to the power of $m$ we obtain
\[(ab)^{qm} = a^{qm} b^{qm} = a^{qm} e^q = a^{qm} = e.\]
From this and Theorem 21.2 (ii) we know that $n | qm$ and because $m$ and $n$ are coprime $n | q$.
Raising both sides of equation \eqref{181} to the power of $n$ we obtain
\[(ab)^{qn} = a^{qn} b^{qn} = e^q b^{qn} = b^{qn} = e.\]
From this and Theorem 21.2 (ii) we know that $m | qn$ and because $m$ and $n$ are coprime $m | q$.
Because $n | q$, $m | q$ and $m$ and $n$ are coprime we can conclude that $mn | q$. We have already shown that $q | mn$, and so we can conclude that $mn = q$. In other words, the order of $ab$ is equal to the product of the orders of $a$ and $b$.
\end{proof}

(2) Let $n$ be an integer with $n \geq 3$.\\
(a) If $n$ is even, show that the center of $D_n$ is not trivial. Then find all of the elements in $Z(D_n)$.\\
(b) If $n$ is odd, find all elements in $Z(D_n)$.\\
~\\
(10) Let $n$ be an integer greater than $2$. Prove that the center of $s_n$ is $\{I \}$, where $I$ is the identity permutation in $S_n$.\\
~\\
(12) When is the cycle $(a_1 a_2 \cdots a_k)$ in $S_n$ even and when is it odd?\\
When $k$ is an even integer greater than or equal to zero 






\end{document}