\documentclass[11pt,a4paper]{article}

\usepackage{amsmath}
\usepackage{amsfonts}
\usepackage{amssymb}
\usepackage{amsthm}

\usepackage{graphicx}

\usepackage{verbatim}

\usepackage{hyperref}


%no paragraph indent
\setlength{\parindent}{0pt}


\begin{document}

\begin{flushright}
Russ Johnson\\
Problem Set $\#4$\\
\today\\
\end{flushright}

(4)\\
{ (a)} Show that $U_{22}$ is cyclic.\\
~\\
The congruence class $[7]$ is a generator of $U_{22}$ and therefore $U_{22}$ is cyclic.
\[U_{22} = \langle [7] \rangle = \{[7], [5],[13],[3],[21],[15],[17],[9],[19],[1]\}.\]
~\\
(b) Find all the generators of $U_{22}$. Explain how you know that each element is a generator.\\
~\\
We know that $U_{22}$ is a group and therefore is closed under its operation. Therefore, by finding an element in $U_{22}$ with the same order as $U_{22}$, we are guaranteed that this is a generator of $U_{22}$. The order of $U_{22}$ is $10$ and from part $(a)$ we already have the generator $[7]$ with order $10$. From Theorem 21.3 part (ii) we know that the order of $[7]^k$ for some positive integer $k$ is equal to $\dfrac{10}{gcd(k,10)}$. Therefore, when $k$ and $10$ are coprime, we know that the order of $[7]^k$ is $10$ and is a generator of $U_{22}$. And so, the generators of $U_{22}$ are $[7]^3 = [13]$, $[7]^7 = [17]$, and $[7]^9 = [19]$.\\
~\\
{(5)} Let $A = \begin{bmatrix}1&0\\0&-1\end{bmatrix} $ and $B = \begin{bmatrix}1&1\\0&-1\end{bmatrix}$.\\
{(a)} Find $|A|$ and $|B|$.\\
~\\
We note that $\langle A \rangle =\left\{ \begin{bmatrix}1&0\\0&-1\end{bmatrix}, \begin{bmatrix}1&0\\0&1\end{bmatrix} \right\}$ and  $\langle B \rangle = \left\{ \begin{bmatrix}1&1\\0&-1\end{bmatrix}, \begin{bmatrix}1&0\\0&1\end{bmatrix} \right\}$. And so, $|A| = |B| = 2$.
~\\
(b) Determine $|AB|$. Does your answer surprise you? Explain.\\
~\\
First of all,
\[AB = \begin{bmatrix}1&1\\0&1\end{bmatrix}, \]
\[AB^2 = \begin{bmatrix}1&2\\0&1\end{bmatrix},\]
and
\[AB^3 = \begin{bmatrix}1&3\\0&1\end{bmatrix}.\]
From this we see that \[\langle AB \rangle = \left\{ \begin{bmatrix}1&a\\0&1\end{bmatrix} | a \in \mathbb{N}   \right\}.\]
And so, $AB$ has infinite order. This answer is somewhat surprising, but by multiplying $A$ and $B$ we obtain a matrix that is not in $\langle A \rangle$ or $\langle B \rangle$ and so we cannot expect it to be finite.\\
~\\
(9) Prove Theorem 21.5.
\begin{proof}
Let $G = \langle a \rangle$ be an infinite cyclic group with identity $e$, and let $b \neq e$ be an element in $G$. Let $m,n\in \mathbb{Z}^+$ such that 
\begin{equation}\label{91}
b^m = b^n.
\end{equation}
The element $a$ is the generator of $G$ and therefore there must exist some positive integer $k$ such that 
\begin{equation}\label{92}
a^k = b.
\end{equation}
Raising both sides of equation \eqref{92} to the power of $m$ we obtain
\begin{equation}\label{93}
a^{km} = b^m.
\end{equation}
Raising both sides of equation \eqref{92} to the power of $n$ we obtain
\begin{equation}\label{94}
a^{kn} = b^n.
\end{equation}
From equations \eqref{91}, \eqref{93}, and \eqref{94} we know that
\begin{equation}\label{95}
a^{kn} = a^{km}.
\end{equation}
The element $a$ has infinite order and so from \eqref{95} we know that
\begin{equation}\label{96}
kn = km.
\end{equation}
Applying the Group Cancellation Rule to \eqref{96} we obtain
\[n = m.\]
In conclusion, we have shown that if $b^m = b^n$ for some positive integer powers $m$ and $n$, then $m = n$. The contrapositive of this is that all integer powers of $b$ are distinct. And so by proving this fact we have shown that $\langle b \rangle$ is an infinite cyclic group. The fact that $\langle b \rangle$, is a cyclic group comes from the Theorem 21.1 and the unique powers proof means that $b$ has infinite order.
\end{proof}
~\\
(18) Let $G$ be a group and let $a,b\in G$ with $|a| = b$ and $|b| = m$.\\
(a) Is it necessarily true that $|ab| = mn$?\\
(b) If $ab = ba$, is it necessarily true that $|ab| = mn$?\\
~\\
This counter-example works for parts (a) and (b). The congruence classes $[7]$ and $[7]$ are both in $U_{22}$. We have already shown that $[7]$ has an order of $10$ and $U_{22}$ also has an order of $10$. It is impossible for an element in $U_{22}$ to have an order of $100$, which is greater than the order of the group and so it is not necessarily true that $|ab| = mn$ when the elements commute.\\
~\\
(c) Prove that if $ab = ba$ and $gcd(m,n) = 1$, then the order of $ab$ is $mn$.

\begin{proof}
First of all we have must show that $ab$ has finite order. We know that $ab = ba$ and so
\[(ab)^{nm} = a^{nm}b^{nm} = (a^n)^m (b^m)^n = e^m e^n = e.\]
From this we see that $ab$ does have finite order. Let $q$ be the order of $ab$. From Theorem 21.2 (ii) we know that $q | nm$. From Theorem 21.2 (i) we know that
\begin{equation}\label{181}
(ab)^q = e.
\end{equation}
Raising both sides of equation \eqref{181} to the power of $m$ we obtain
\[(ab)^{qm} = a^{qm} b^{qm} = a^{qm} e^q = a^{qm} = e.\]
From this and Theorem 21.2 (ii) we know that $n | qm$ and because $m$ and $n$ are coprime $n | q$.
Raising both sides of equation \eqref{181} to the power of $n$ we obtain
\[(ab)^{qn} = a^{qn} b^{qn} = e^q b^{qn} = b^{qn} = e.\]
From this and Theorem 21.2 (ii) we know that $m | qn$ and because $m$ and $n$ are coprime $m | q$.
Because $n | q$, $m | q$ and $m$ and $n$ are coprime we can conclude that $mn | q$. We have already shown that $q | mn$, and so we can conclude that $mn = q$. In other words, the order of $ab$ is equal to the product of the orders of $a$ and $b$.
\end{proof}

(2) Let $n$ be an integer with $n \geq 3$.\\
(a) If $n$ is even, show that the center of $D_n$ is not trivial. Then find all of the elements in $Z(D_n)$.\\
(b) If $n$ is odd, find all elements in $Z(D_n)$.\\
Let $R$ be the smallest rotation in $D_n$ and let $r$ be any reflection in $D_n$. We know that any element of $D_n$ can be written as a power of $R$ or $r$ times a power of $R$. In other words
\[D_n = \{ R^k | 0 \leq k < n \} \cup \{ rR^k | 0 \leq k < n \}.\]
We are looking for elements in $D_n$ that commute with all other elements in $D_n$. In other words, $x\in D_n$ is in the center of $D_n$ if and only if $ax = xa$ for all $a \in D_n$.\\
~\\
Assume that $p$ be an integer such that $0 \leq p < n$ and $rR^p$ is in the center of $D_n$. We will prove that this leads to a contradiction. Because $rR^p$ is in the center of $D_n$
\begin{equation}\label{a1}
(rR^p)R^k = R^k(rR^p)
\end{equation}
for all $k\in \mathbb{Z}$ such that $0 \leq k < n$. Applying the associative property to \eqref{a1} we obtain
\begin{equation}
(rR^p)R^k = (R^kr)R^p
\end{equation}
Next we substitute $R^kr$ with $rR^{-k}$ coming from the presentation of $D_n$
\begin{equation}\label{a3}
(rR^p)R^k = (rR^{-k})R^p.
\end{equation}
Next we apply the associative property to \eqref{a3} to obtain
\begin{equation}\label{a4}
(rR^p)R^k = r(R^{-k}R^p).
\end{equation}
Next we commute the rotations on the right side of \eqref{a4} to obtain
\begin{equation}\label{a5}
(rR^p)R^k = r(R^pR^{-k}).
\end{equation}
Next we apply the associative property to \eqref{a5} to obtain
\begin{equation}\label{a6}
(rR^p)R^k = (rR^p)R^{-k}.
\end{equation}
Finally we apply the group cancellation law to \eqref{a6} and arrive at
\[R^k = R^{-k}.\]
The inverse of a rotation is not always the rotation itself, except when we are working with $n\leq 2$. Since we are working with $n \geq 3$ we can conclude that there is no such integer $p$ such that $rR^p$ is in the center of $D_n$ when $n \geq 3$.\\
~\\
Next we assume that $p$ is an integer such that $0 \leq p < n$ and $R^p$ is in the center of $D_n$. Because $R^p$ is in the center of $D_n$,
\[R^pR^k = R^kR^p\]
and
\[R^p(rR^k) = (rR^k)R^p\]
for all $k\in \mathbb{Z}$ such that $0 \leq k < n$. The first equation is true for any $p$, because rotations commute with each other. For the second equation we first apply the associative property to the right side of the equation to obtain
\begin{equation}\label{21}
R^p(rR^k) = r(R^kR^p).
\end{equation}
Next we apply the knowledge that rotations commute with one other to the right side of equation \eqref{21} to obtain
\begin{equation}\label{22}
R^p(rR^k) = r(R^pR^k).
\end{equation}
Applying the associative property to the right side of \eqref{22}, we obtain
\begin{equation}\label{23}
R^p(rR^k) = (rR^p)R^k.
\end{equation}
Next we substitute $rR^p$ with $R^{-p}r$ coming from the presentation of $D_n$
\begin{equation}\label{24}
R^p(rR^k) = (R^{-p}r)R^k.
\end{equation}
We then apply the associative property to \eqref{24} to obtain
\begin{equation}\label{25}
R^p(rR^k) = R^{-p}(rR^k).
\end{equation}
Finally, applying the group cancellation law to \eqref{25} we arrive at
\[R^p = R^{-p}.\]
This is true when $p = 0$ regardless of whether $n$ is even or odd. In this case, $R^0 = R^{-0} = I$. The other case when this is possible is when $p = n/2$. When $n$ is odd $n/2$ is not an integer, but when $n$ is even $n/2$ is an integer.\\
~\\
In conclusion, when $n$ is odd $Z(D_n) = \{I \}$ and when $n$ is even $Z(D_n) = \{I, R^{n/2}\}$.\\
~\\
(10) Let $n$ be an integer greater than $2$. Prove that the center of $S_n$ is $\{I \}$, where $I$ is the identity permutation in $S_n$.\\
\begin{proof}
We know from its definition that the identity $I$ of $S_n$ commutes with all elements of this group and so $I\in Z(S_n)$. Now we will prove that there is no other element in the center of $S_n$. Let $p\in S_n$ not equal to the identity of $S_n$. Because $p$ is not the identity, we know that there exist distinct points $i$ and $j$ such that $p(i) = j$. Because $n\geq 3$, there exists some $q\in S_n$ such that $q(j) = k$ and $q(k) = j$ with $k\neq i$ and $k\neq j$ and fixes everything else. The permutation $q$ fixes the point $i$ and so $q^{-1}$ must also fix $i$. And so,
\[qpq^{-1}(i) = qp(i) = q(j) = k.\]
If the permutation $p$ commuted with all other elements in $S_n$, we would have
\[qpq^{-1}(i) = qq^{-1}p(i) = j.\]
Since we know that $k\neq j$, we can conclude that the permutation $p$ does not commute with all other elements in $S_n$. The only condition that we placed on the permutation $p$ was that it is not the identity, and so the only element in $Z(S_n)$ is $I$.
\end{proof}
(12) When is the cycle $(a_1 a_2 \cdots a_k)$ in $S_n$ even and when is it odd?\\
~\\
{\bf Conjecture.} When $k$ is an even integer such that $k\geq 1$ the cycle $(a_1 a_2 \cdots a_k)$ in $S_n$ is odd and when $k$ is an odd integer such that $k\geq 1$ the cycle $(a_1 a_2 \cdots a_k)$ in $S_n$ is even.
\begin{proof}
Let $k=2$. The cycle $(a_1 a_2)$ is itself a single transposition and therefore is odd. From this we have our base step for a proof by induction. Next assume that $k$ is even and $(a_1 a_2 \cdots a_k)$ is an odd cycle. We will prove that $(a_1 a_2 \cdots a_k a_{k+1} a_{k+2})$ is an odd cycle. Because $(a_1 a_2 \cdots a_k)$ is an odd cycle, we also know that its factorization $(a_1a_k)(a_1a_{k-1})\cdots(a_1a_2)$ is made up of an odd number of transpositions. The cycle $(a_1 a_2 \cdots a_k a_{k+1} a_{k+2})$ can be decomposed as $(a_1a_{k+2})(a_1a_{k+1})(a_1a_k)(a_1a_{k-1})\cdots(a_1a_2)$ which has contains exactly two more transpositions than $(a_1a_k)(a_1a_{k-1})\cdots(a_1a_2)$ and therefore $(a_1 a_2 \cdots a_k a_{k+1} a_{k+2})$ is an odd cycle. By induction we have shown that if $k$ is a positive even integer then the cycle $(a_1 a_2 \cdots a_k)$ is odd.\\
~\\
Next we let $k=1$. This gives us the identity which we know to be an even cycle. From this we have our base step for a proof by induction. Next assume that $k$ is odd and $(a_1 a_2 \cdots a_k)$ is an even cycle. We will prove that $(a_1 a_2 \cdots a_k a_{k+1} a_{k+2})$ is an even cycle. Because $(a_1 a_2 \cdots a_k)$ is an even cycle, we also know that its factorization $(a_1a_k)(a_1a_{k-1})\cdots(a_1a_2)$ is made up of an even number of transpositions. The cycle $(a_1 a_2 \cdots a_k a_{k+1} a_{k+2})$ can be decomposed as $(a_1a_{k+2})(a_1a_{k+1})(a_1a_k)(a_1a_{k-1})\cdots(a_1a_2)$ which has contains exactly two more transpositions than $(a_1a_k)(a_1a_{k-1})\cdots(a_1a_2)$ and therefore $(a_1 a_2 \cdots a_k a_{k+1} a_{k+2})$ is an even cycle. By induction we have shown that if $k$ is a positive odd integer then the cycle $(a_1 a_2 \cdots a_k)$ is even.\\
\end{proof}

\end{document}