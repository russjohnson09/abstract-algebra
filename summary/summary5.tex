\documentclass[11pt,a4paper]{article}

\usepackage{amsmath}
\usepackage{amsfonts}
\usepackage{amssymb}

\usepackage{enumitem}

\begin{document}
\begin{flushright}
Russ Johnson\\
Reading Assignment 5\\
\today\\
\end{flushright}
For activities 20.7 and 20.8 we looked at the smallest subgroup of a group containing an element in the group. For activity 20.7 we looked at the smallest subgroup of the group $(\mathbb{Z},+)$ containing $5$. This subgroup must contain the identity element zero, the inverse of $5$, and any element that can be obtained using addition of these three elements (closure property of a group). We proved that the set $\{5m|m\in\mathbb{Z}\}$ is a subset of this smallest subgroup and went on to prove that the smallest subgroup of $(\mathbb{Z},+)$ is a subset of $\{5m|m\in\mathbb{Z}\}$ and is therefore equal to $\{5m|m\in\mathbb{Z}\}$.

In activity 20.8 we generalized what we had done in 20.7 and showed that the smallest subgroup of $G$ containing $a$, denoted $\langle a\rangle$, is the set $\{a^n|n\in\mathbb{Z}\}$. In this case, $a^0$ is the identity element in $G$ and $a^{-n}$, where $n\in\mathbb{N}$, is the inverse of $a^n$. 
\end{document}

Guidelines:
1. The statement of each problem must be written in its entirety, followed by the solution.

2. If you obtain an idea from someone else credit them.

3. Write in complete sentences.

4. Avoid run-ons.

5. Do not begin a sentence with a symbol.

6. As a writer, it is your job to make things clear to the reader. Do not use the words "clearly" or obviously.

7. Proofread. Use a spell-checker and be sure to re-read everything at least two times for accuracy and clarity before submitting.


Understand your audience.

Steve Schlicker