\documentclass[11pt,a4paper]{article}

\usepackage{amsmath}
\usepackage{amsfonts}
\usepackage{amssymb}

\begin{document}
\begin{flushright}
Russ Johnson\\
Reading Assignment 11\\
\today\\
\end{flushright}
~\\
~\\
For the proof of Lemma 23.8 we start out by showing that there are at least two transpositions in the transposition decomposition of $I$. From here we go on to prove that if we have two or more transpositions in the decomposition of $I$, the total number of transpositions is even. In this proof we show that if the number of transpositions is odd we are able to break the decomposition down until we reach a contradiction where some transposition is not equal to itself.\\
~\\
Lemma 23.8 makes since as a starting point for Theorem 23.8, because it proves that the most basic permutation has a constant parity. The permutation $I$ can be written simply as a transposition multiplied by itself. From Theorem 23.8 this means that all transposition decompositions of $I$ must have an even parity. By proving Lemma 23.8, Theorem 23.8 becomes much easier to prove.
\end{document}