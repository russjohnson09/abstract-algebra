\documentclass[11pt,a4paper]{article}

\usepackage{amsmath}
\usepackage{amsfonts}
\usepackage{amssymb}

\begin{document}
\begin{flushright}
Russ Johnson\\
Reading Assignment 17\\
\today\\
\end{flushright}
~\\
~\\
In parts $(f)$ and $(g)$ of Activity 27.11 we show that group $G$ has the same presentation as $D_p$. In doing so, we prove that the group $D_p$ is isomorphic to $G$. In group $G$, $a$ plays the role of $r$ in $D_p$ and $b^i$ plays the role of $R$ in $D_p$. We started the activity with an arbitrary non-Abelian group $G$ with order $p$ where $p$ is an odd prime. We then proved that this group is isomorphic with $D_p$. In this way, we proved that $G\cong D_p$. Therefore, Theorem 27.5 is valid.
\end{document}