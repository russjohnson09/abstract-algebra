\documentclass[11pt,a4paper]{article}

\usepackage{amsmath}
\usepackage{amsfonts}
\usepackage{amssymb}

\begin{document}
\begin{flushright}
Russ Johnson\\
Reading Assignment 9\\
\today\\
\end{flushright}
~\\
~\\
For Activity 23.1 we showed that the permutations of some set $S$ is closed under composition. We did this by showing that the composition of two bijective functions is also a bijective function. Because a permutation is defined as a bijection on the some set $S$, this proves that the composition of two permutations of some set is also a permutation of this set. I think that one focus of this activity is to draw a connection between symmetries and permutations. For the last part of this problem we proved that each permutation on some set has an inverse. All that we have left to show is that the associative property holds and their is an identity. The identity would simply be the permutation where $f(x) = x$ for all $x\in S$. The associativity property holds, because it is a property of the operator composition of functions.\\
~\\
And so, from this activity we see that the set of permutations of some set $S$ under composition of functions is a group. The main focus of this activity was to work out a proof of this statement.
\end{document}