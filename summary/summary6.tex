\documentclass[11pt,a4paper]{article}

\usepackage{amsmath}
\usepackage{amsfonts}
\usepackage{amssymb}

\usepackage{enumitem}

\begin{document}
\begin{flushright}
Russ Johnson\\
Reading Assignment 5\\
\today\\
\end{flushright}
~\\
~\\

For Activity 21.1 we looked at the nontrivial subgroup $H$ of the cyclic group $\mathbb{Z}_{100}$. The smallest number of elements of the subgroup $H$ containing $[20]$ is $5$. In this case the subgroup $H=\{[0],[20],[40],[60],[80]\}$. It should be noted that $5[20]=[0]$. For all finite cyclic groups equal to $\left\langle a \right\rangle$ with identity $e$ and cardinality $n$, $na=e$. This is true of $H$, because it is also a cyclic group. We will see later that all subgroups of cyclic groups are cyclic groups themselves. There are other subgroups of $\mathbb{Z}_{100}$ containing $[20]$ besides the group with cardinality $5$. One example is the group $\{[0],[10],[20],[30],[40],[50],[60],[70],[80],[90]\}$. This subgroup of $\mathbb{Z}_{100}$ is equal to $\langle [10]\rangle$ and $[20]$ is divisible by $[10]$. There is also the subgroup of $\mathbb{Z}_{100}$ equal to $\mathbb{Z}_{100}$, which will also contain $[20]$. Overall, the goal of this activity is to look at subgroups of cyclic groups and the properties that these subgroups have. Describing all subgroups of a group is sometimes difficult, but cyclic groups have simpler subgroup structures and can be investigated more throughly than other subgroups.

\end{document}