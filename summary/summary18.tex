\documentclass[11pt,a4paper]{article}

\usepackage{amsmath}
\usepackage{amsfonts}
\usepackage{amssymb}

\begin{document}
\begin{flushright}
Russ Johnson\\
Reading Assignment 17\\
\today\\
\end{flushright}
~\\
~\\
Theorem 27.4 states that group isomorphism is an equivalence relation. This means that a group is isomorphic with itself, if group $A$ is isomorphic with group $B$, then $B$ is isomorphic with $A$, and if $A$ is isomorphic with $B$ and $B$ with $C$, then group $A$ is isomorphic with $B$. So, in order to prove that group isomorphism is an equivalence relation we must prove that it has the properties of an equivalence relation. First we prove that group $A$ is isomorphic with itself. Its isomorphism is $\phi: A \rightarrow A$ such that $\phi(x) = x$ for all $x\in A$. We proved that this identity map is bijective and it preserves the structure of the group as we would expect from an identity in exercise (1). The next part of the proof is to proof that group isomorphism is symmetric. We assume that we have an isomorphism from group $A$ to group $B$ and from this we define a function that does its opposite to create an isomorphism from $B$ to $A$. We first must prove that $\phi^{-1}$ is in fact a function. After this we prove that it is bijective and that it preserves structure. Finally, we must prove the transitive property of isomorphism.  We do this in a similar way to proving the symmetric property. We construct a isomorphism from group $A$ to group $C$ using the isomorphisms from $A$ to $B$ and $B$ to $C$. Again, we must prove that the function is well defined, is bijective, and preserves structure.
\end{document}